\documentclass{jsarticle}
\usepackage[dvipdfmx]{graphicx}
\usepackage{amsmath}
\usepackage{color}
\usepackage{colortbl}
\usepackage{arydshln}

\newcommand*{\mbold}[1]{\mbox{\boldmath $#1$}}

\renewcommand*{\labelenumi}{(\arabic{enumi})}

\begin{document}
\section{Preparation of Mathmatics}
\subsection{Vector product}
\subsubsection*{Theorem1.02}\label{Theorem1.02}
\begin{enumerate}
  \item $\mbold{a} \cdot \mbold{a} = \mbold{0}$ \\
    ただそのまま計算するだけ. \\
    \[
      \begin{pmatrix}
        a_1 \\
        a_2 \\
        a_3
      \end{pmatrix}
      \times
      \begin{pmatrix}
        a_1 \\
        a_2 \\
        a_3
      \end{pmatrix}
      =
      \begin{pmatrix}
        a_2 a_3 - a_3 a_2 \\
        a_3 a_1 - a_1 a_3 \\
        a_1 a_2 - a_2 a_1
      \end{pmatrix}
      = \mbold{0}
    \]
  \item $\mbold{a} \times \mbold{b} \perp \mbold{a}$, $\mbold{a} \times \mbold{b} \perp \mbold{b}$ \\
    ただそのまま計算するだけ. \\
    \[
      \begin{pmatrix}
        a_2 b_3 - a_3 b_2 \\
        a_3 b_1 - a_1 b_3 \\
        a_1 b_2 - a_2 b_1
      \end{pmatrix}
      \cdot
      \begin{pmatrix}
        a_1 \\
        a_2 \\
        a_3
      \end{pmatrix}
      = a_1 a_2 b_3 - \textcolor{red}{a_1 a_3 b_2} + \textcolor{blue}{a_2 a_3 b_1} - a_1 a_2 b_3 + \textcolor{red}{a_1 a_3 b_2} - \textcolor{blue}{a_2 a_3 b_1} = 0 
    \]
    \[
      \begin{pmatrix}
        a_2 b_3 - a_3 b_2 \\
        a_3 b_1 - a_1 b_3 \\
        a_1 b_2 - a_2 b_1
      \end{pmatrix}
      \cdot
      \begin{pmatrix}
        b_1 \\
        b_2 \\
        b_3
      \end{pmatrix}
      = b_1 a_2 b_3 - \textcolor{red}{b_1 a_3 b_2}
      + \textcolor{red}{b_2 a_3 b_1} - \textcolor{blue}{b_2 a_1 b_3}
      + \textcolor{blue}{b_3 a_1 b_2} - b_3 a_2 b_1 = 0 
    \]
  \item $\mbold{a}$と$\mbold{b}$で張る平行四辺形の面積$S$に対し, $S = |\mbold{a} \times \mbold{b}|$ \\
    $\mbold{a}$と, $\mbold{b}$の成す角度を$\theta$とすると, 
    \[
      S = |\mbold{a}||\mbold{b}||\sin\theta|
      = |\mbold{a}||\mbold{b}|\sqrt{1 - \cos^2\theta}
      = \sqrt{|\mbold{a}|^2 |\mbold{b}|^2 - (\mbold{a}\cdot\mbold{b})^2}
    \]
    成分表示すると, 
    \[
      \sqrt{|\mbold{a}|^2 |\mbold{b}|^2 - (\mbold{a}\cdot\mbold{b})^2}
      = \sqrt{(\sum_{i = 1}^{3} a_i^2) (\sum_{i = 1}^{3} b_i^2) - (\sum_{i = 1}^{3} a_i b_i)^2}
    \]
    ルート内の第一項は, 多項式の展開であり, 添字が同じ項と, 異なる項にわけられる. 
    \[
      (\sum_{i = 1}^{3} a_i^2) (\sum_{i = 1}^{3} b_i^2) = \sum_{i,j=1}^{3}a_i^2 b_j^2 = \sum_{i = 1}^{3}a_i^2 b_i^2 + \sum_{i \neq j}a_i^2 b_j^2
    \]
    第二項も同様に展開でき, 
    \[
      (\sum_{i = 1}^{3} a_i b_i)^2 = \sum_{i,j = 1}^{3}a_i b_i a_j b_j = \sum_{i = 1}^{3}a_i^2 b_i^2 + \sum_{i\neq j}a_i a_j b_i b_j
    \]
    以上により, 
    \[
      S^2 = \sum_{i \neq j}a_i^2 b_j^2 - \sum_{i \neq j}a_i a_j b_i b_j
    \]
    第一項, 第二項ともに, $i > j$のときと, $i < j$のときに分けることができ, 
    \[
      \sum_{i \neq j}a_i^2 b_j^2 - \sum_{i \neq j}a_i a_j b_i b_j
      = \sum_{i > j}a_i^2 b_j^2 + \sum_{i < j}a_i^2 b_j^2 - \sum_{i > j}a_i a_j b_i b_j - \sum_{i < j}a_i a_j b_i b_j
    \]
    各4項のインデックス$i, j$は, 項の中での計算でクローズしているものであり, 文字を書き換えたとしても, 計算結果に影響を与えない. そこで, 第二項, 及び, 第四項について, $i \leftrightarrow j$の入れ替えを実施し, 
    \[
      \sum_{i > j}a_i^2 b_j^2 + \sum_{i < j}a_i^2 b_j^2 - \sum_{i > j}a_i a_j b_i b_j - \sum_{i < j}a_i a_j b_i b_j
      = \sum_{i > j}a_i^2 b_j^2 + \sum_{i > j}a_j^2 b_i^2 - \sum_{i > j}a_i a_j b_i b_j - \sum_{i > j}a_j a_i b_j b_i
    \]
    を得る. インデックス$i, j$, 及び和条件$i > j$が揃っているので, この和を一つにまとめると, 
    \[
      \sum_{i > j}a_i^2 b_j^2 + a_j^2 b_i^2 - a_i a_j b_i b_j - a_j a_i b_j b_i = \sum_{i > j}a_i^2 b_j^2 + a_j^2 b_i^2 - 2 a_i a_j b_i b_j = \sum_{i > j}(a_i b_j - a_j b_i)^2
    \]
    となる. 
    和の中の, $i = 3, j = 2$を考えると, $(a_3 b_2 - a_2 b_3)^2 = (a_2 b_3 - a_3 b_2)^2 = (\mbold{a} \times \mbold{b})_z^2$となっており, 同様に, $i = 3, j = 1$で$y$成分, $i = 2, j = 1$で, $x$成分になっているため, これは, $|\mbold{a} \times \mbold{b}|^2$にほかならない. 

\subsubsection*{Formula1.03}
\[
  \mbold{a} \cdot (\mbold{b} \times \mbold{c}) = \mbold{b} \cdot (\mbold{c} \times \mbold{a}) = \mbold{c} \cdot (\mbold{a} \times \mbold{b})
\]
これらの式は, $\mbold{a}$, $\mbold{b}$, $\mbold{c}$で張られる平行六面体の体積
最初の式について成分表示すると, 
\[
  \mbold{a} \cdot (\mbold{b} \times \mbold{c}) = 
  \begin{pmatrix}
        a_1 \\
        a_2 \\
        a_3
  \end{pmatrix}
  \cdot
  \begin{pmatrix}
    b_2 c_3 - b_3 c_2 \\
    b_3 c_1 - b_1 c_3 \\
    b_1 c_2 - b_2 c_1
  \end{pmatrix}
  = a_1 b_2 c_3 - a_1 b_3 c_2 + a_2 b_3 c_1 - a_2 b_1 c_3 + a_3 b_1 c_2 - a_3 b_2 c_1
\]
この式について, 第二式, 第三式を実現するような, $a, b, c$の単純な文字の置き換えで, 式が変化しないかどうかを確認する. 
例えば, 第二式の場合は, $a \rightarrow b$, $b \rightarrow c$, $c \rightarrow a$という置き換えであるが, 第一項について, その置き換えをすると, $b_1 c_2 a_3 = a_3 b_1 c_2$となり, 第5項と一致する. これを, 他の項についても以下のような表を用いて確認していく. 
\begin{table}[hbtp]
  \caption{第一式の添字}
  \centering
  \begin{tabular}{|c|c|c|c|}
    \hline
    a & b & c & sgn \\
    \hline \hline
    1 & 2 & 3 & + \\
    \hdashline
    \rowcolor{yellow}1 & 3 & 2 & - \\
    \hdashline
    \rowcolor{green}2 & 3 & 1 & + \\
    \hdashline
    \rowcolor{red}2 & 1 & 3 & - \\
    \hdashline
    \rowcolor{cyan}3 & 1 & 2 & + \\
    \hdashline
    \rowcolor{magenta}3 & 2 & 1 & - \\
    \hline
  \end{tabular}
\end{table}

\begin{table}[hbtp]
  \caption{第二式に対応した入れ替えによる添字}
  \centering
  \begin{tabular}{|c|c|c|c|}
    \hline
    a & b & c & sgn \\
    \hline \hline
    \rowcolor{green}2 & 3 & 1 & + \\
    \hdashline
    \rowcolor{magenta}3 & 2 & 1 & - \\
    \hdashline
    \rowcolor{cyan}3 & 1 & 2 & + \\
    \hdashline
    \rowcolor{yellow}1 & 3 & 2 & - \\
    \hdashline
    1 & 2 & 3 & + \\
    \hdashline
    \rowcolor{red}2 & 1 & 3 & - \\
    \hline
  \end{tabular}
\end{table}
\begin{table}[hbtp]
  \caption{第三式に対応した入れ替えによる添字}
  \centering
  \begin{tabular}{|c|c|c|c|}
    \hline
    a & b & c & sgn \\
    \hline \hline
    \rowcolor{cyan}3 & 1 & 2 & + \\
    \hdashline
    \rowcolor{red}2 & 1 & 3 & - \\
    \hdashline
    1 & 2 & 3 & + \\
    \hdashline
    \rowcolor{magenta}3 & 2 & 1 & - \\
    \hdashline
    \rowcolor{green}2 & 3 & 1 & + \\
    \hdashline
    \rowcolor{yellow}1 & 3 & 2 & - \\
    \hline
  \end{tabular}
\end{table}
表を比べると, ただ入れ替えただけの結果が, 確かに符号まで含めて, 色付けしたところ同士で同じ項になっており, 恒等式になっている. 
\\

$\mbold{a} \cdot (\mbold{b} \times \mbold{c})$について考える. Theorem1.02によれば, $\mbold{b} \times \mbold{c}$は, $\mbold{b}$と$\mbold{c}$で張られる平行四辺形の面積$S$の大きさを持ち, その面に対する法線の方向$\mbold{n}$のベクトルであった. 
したがって
\[
  \mbold{a} \cdot (\mbold{b} \times \mbold{c}) = S \mbold{a} \cdot \mbold{n}
\]
右辺は, $\mbold{a}$の$\mbold{n}$方向成分の符号付き長さであり, 件の平行六面体においては, $S$を底面とするなら, 高さに相当する. 
したがって, これは体積を表す. 
\end{enumerate}
\end{document}
