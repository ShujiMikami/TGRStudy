\documentclass{jsarticle}
\usepackage[dvipdfmx]{graphicx}
\usepackage[dvipdfmx]{hyperref}
\usepackage{amsmath}
\usepackage{color}
\usepackage{colortbl}
\usepackage{arydshln}
\usepackage{mathtools}

\newcommand*{\mbold}[1]{\mbox{\boldmath $#1$}}

%\renewcommand*{\labelenumi}{(\arabic{enumi})}

\newcommand*{\transp}[1]{\prescript{t\!}{}{#1}}

\newcommand*{\grad}{{\rm grad}}
\newcommand*{\divg}{{\rm div}}
\newcommand*{\rot}{{\rm rot}}
\newcommand*{\trace}[1]{{\rm tr}\!{#1}}


\title{Formula1.18}

\begin{document}
\maketitle

\begin{abstract}
  \begin{equation}
    \divg(f(\mbold{x})\mbold{A}(\mbold{x})) = 
    f(x)\divg\mbold{A}(\mbold{x}) + \grad f(\mbold{x}) \cdot \mbold{A}(\mbold{x})
  \end{equation}
\end{abstract}

\section*{証明}
定義通り計算するだけ. 関数の引数は省略する. 
\begin{eqnarray}
  && \divg(f\mbold{A}) \nonumber \\
  && = \sum_i \frac{\partial (f A_i)}{\partial x_i} \nonumber \\
  && = \sum_i (\frac{\partial f}{\partial x_i} A_i + f \frac{\partial A_i}{\partial x_i}) \nonumber \\
  && = \sum_i (\grad f)_i A_i + f \divg\mbold{A} \nonumber \\
  && = \grad f \cdot \mbold{A} + f \divg\mbold{A}
\end{eqnarray}


\end{document}
