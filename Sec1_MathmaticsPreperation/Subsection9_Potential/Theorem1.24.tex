\documentclass{jsarticle} \usepackage[dvipdfmx]{graphicx} \usepackage[dvipdfmx]{hyperref}
\usepackage{amsmath}
\usepackage{color}
\usepackage{colortbl}
\usepackage{arydshln}
\usepackage{mathtools}

\newcommand*{\mbold}[1]{\mbox{\boldmath $#1$}}

%\renewcommand*{\labelenumi}{(\arabic{enumi})}

\newcommand*{\transp}[1]{\prescript{t\!}{}{#1}}

\newcommand*{\grad}{{\rm grad}}
\newcommand*{\divg}{{\rm div}}
\newcommand*{\rot}{{\rm rot}}
\newcommand*{\trace}[1]{{\rm tr}\!{#1}}


\title{Theory1.24}

\begin{document}
\maketitle

\begin{abstract}
  ベクトル場$\mbold{A}(\mbold{x})$が, 直方体の領域$R$全体で, $\divg \mbold{A}(\mbold{x}) = \mbold{0}$を満たすとき, ベクトル場$\mbold{B}(\mbold{x})$が存在し, 
  $\mbold{A}(\mbold{x}) = \rot \mbold{B}(\mbold{x})$と表される. 
  $\mbold{B}(\mbold{x})$をベクトルポテンシャルと呼ぶ
\end{abstract}

\section*{本証明}
ベクトルポテンシャル$\mbold{B}$が存在するという状況, すなわち, 
\begin{equation}
  \mbold{A} = \rot\mbold{B}
\end{equation}
が成立するとは, すなわち, $\rot$の逆演算を$\mbold{A}$に施した結果が値を持つこと, と読み替えられる. 
この手の処理は, 一般的な解放はなく, ある程度予測をつけて, 微修正していくことになる.  
単純に, 微分処理を逆転させてみる. 
すなわち, 
\begin{subequations}
  \begin{eqnarray}
    && B_x = \int_{y_0}^y A_z dy - \int_{z_0}^z A_y dz \\
    && B_y = \int_{z_0}^z A_x dz - \int_{x_0}^x A_z dx \\
    && B_z = \int_{x_0}^x A_y dx - \int_{y_0}^y A_x dy
  \end{eqnarray}
\end{subequations}
としてみる. これに対し, $\rot$を計算してみると, 
\begin{eqnarray}\label{eq-3}
  && \frac{\partial B_y}{\partial x} - \frac{\partial B_x}{\partial y} \nonumber \\
  && = \frac{\partial}{\partial x}\int_{z_0}^z A_x dz - A_z 
    - A_z + \frac{\partial}{\partial y}\int_{z_0}^z A_y dz \nonumber \\
  && = -2 A_z + \int_{z_0}^z \left(\frac{\partial A_x}{\partial x} + \frac{\partial A_y}{\partial y}\right)dt
\end{eqnarray}
$\divg\mbold{A} = 0$によれば, 
\[
  \frac{\partial A_x}{\partial x} + \frac{\partial A_y}{\partial y} = -\frac{\partial A_z}{\partial z}
\]
なので, 式(\ref{eq-3})は, 積分の自由度を無視すると,  
\begin{equation}
  \frac{\partial B_y}{\partial x} - \frac{\partial B_x}{\partial y}
  = -3 A_z
\end{equation}
同様に, $y$, $z$軸も計算すれば, 
係数$-3$がかかっているが, $\rot\mbold{B} = \mbold{A}$の関係になっている. 
よって, そもそもの定義が, 
\begin{subequations}
  \begin{eqnarray}
    && B_x = \frac{1}{3}\left(\int_{z_0}^z A_y dz - \int_{y_0}^y A_z dy \right)  \\
    && B_y = \frac{1}{3}\left(\int_{x_0}^x A_z dx - \int_{z_0}^z A_x dz \right)  \\
    && B_z = \frac{1}{3}\left(\int_{y_0}^y A_x dy - \int_{x_0}^x A_y dx \right)
  \end{eqnarray}
\end{subequations}
であったなら, 成立している. 
よって, ベクトルポテンシャルの存在は証明できた. 
\end{document}
