\documentclass{jsarticle}
\usepackage[dvipdfmx]{graphicx}
\usepackage{amsmath}
\usepackage{color}
\usepackage{colortbl}
\usepackage{arydshln}

\newcommand*{\mbold}[1]{\mbox{\boldmath $#1$}}

\renewcommand*{\labelenumi}{(\arabic{enumi})}

\title{Theorem1.07}

\begin{document}
\maketitle

\begin{abstract}
  $f(x(t), y(t))$, $f(x(t), y(t), z(t))$の微分
  \begin{equation}\label{eq-1}
    \frac{df}{dt} = \frac{\partial f}{\partial x}\frac{dx}{dt} 
    + \frac{\partial f}{\partial y}\frac{dy}{dt}
  \end{equation}
  \begin{equation}
    \frac{df}{dt} = \frac{\partial f}{\partial x}\frac{dx}{dt} 
    + \frac{\partial f}{\partial y}\frac{dy}{dt}
    + \frac{\partial f}{\partial z}\frac{dz}{dt}
  \end{equation}

  $f(x(u, v), y(u, v))$, $f(x(u, v), y(u, v), z(u, v))$の$u$による偏微分
  \begin{equation}\label{eq-3}
    \frac{\partial f}{du} = \frac{\partial f}{\partial x}\frac{\partial x}{\partial u} 
    + \frac{\partial f}{\partial y}\frac{\partial y}{\partial u}
  \end{equation}
  \begin{equation}
    \frac{\partial f}{\partial u} = \frac{\partial f}{\partial x}\frac{\partial x}{\partial u} 
    + \frac{\partial f}{\partial y}\frac{\partial y}{\partial u}
    + \frac{\partial f}{\partial z}\frac{\partial z}{\partial u}
  \end{equation}
\end{abstract}

\section{独立変数としての計算}\label{sec-1}
$x = x(t)$, $y = y(t)$というのは, $f(x, y)$が定義される, 全ての$(x, y)$が成す点の集合の, 部分集合である. 
したがって, まずは$x$, $y$を独立変数と見て, 微分の計算を行った上で, 部分集合としての縛りを入れていく. 
\begin{equation}\label{eq-5}
  \Delta f = f(x + \Delta x, y + \Delta y) - f(x, y)
\end{equation}
一方, 証明すべき式中の, 偏微分は以下のように定義される. 
\begin{equation}\label{eq-6}
  \frac{\partial f(x, y)}{\partial x} \equiv \lim_{\Delta x \to 0}\frac{f(x + \Delta x, y) - f(x, y)}{\Delta x}
\end{equation}
この式の形を作り込むために, 式(\ref{eq-5})を, 以下のように変形する. 
\begin{equation}\label{eq-7}
  \Delta f = f(x + \Delta x, y + \Delta y) - f(x, y + \Delta y) + f(x, y + \Delta y) - f(x, y)
\end{equation}
式(\ref{eq-7})の第一, 二項は, 式(\ref{eq-6})の右辺分子の, $y$を$y + \Delta y$としたもので, 第三, 四項は, 式(\ref{eq-6})の$y$に関する偏微分の分子そのものになっている. 

偏微分の式の定義は, 極限の意味を考えると, $\Delta x \to 0$において, 右辺の項が, 左辺に収束する, すわなち, ある十分小さな$\Delta x$において, 右辺の項は, 左辺の項に対し, $\Delta x$と同等かそれ以上早く, $0$に収束する項の差しかない, ということであり, 以下のように読み替えられる. 
\begin{equation}\label{eq-8}
  \frac{f(x + \Delta x, y) - f(x, y)}{\Delta x} = \frac{\partial f(x, y)}{\partial x} + O(\Delta x)
\end{equation}
ここは, テイラー展開の知識を少し動員した方が, もう少しエレガントな説明になるかもしれない. 
ちなみに, $O$はランダウ記号と呼ばれ, $o$と, $O$では意味が違う. 
前者は, $\Delta x$と同等の早さで, 後者は, より早く, という意味になる. 
意外とちゃんとわかって使ってなかった. 今回の場合は, ビッグオーである必要がある. 

式(\ref{eq-7})を式(\ref{eq-8})を用いて書き換えると
\begin{equation}\label{eq-9}
  \Delta f
  = \left(\frac{\partial f(x, y + \Delta y)}{\partial x} + O(\Delta x)\right) \Delta x
  + \left(\frac{\partial f(x, y)}{\partial y} + O(\Delta y)\right) \Delta y
  = \frac{\partial f(x, y + \Delta y)}{\partial x} \Delta x + o(\Delta x)
  + \frac{\partial f(x, y)}{\partial y}\Delta y + o(\Delta y)
\end{equation}
ただし, $O(\Delta x)\Delta x$は, 
\[
  \lim_{\Delta x \to 0}\frac{O(\Delta x)\Delta x}{\Delta x} = 0
\]
により, 定義から, スモールオーになる. 

式(\ref{eq-9})の右辺第一項は, $f$が, $(x, y)$について, $C^{\infty}$級であるという前提がある(本の最初の章で, この本で扱う関数はすべて何回でも微分可能, と述べている)ので, 
\[
  \frac{\partial f(x, y)}{\partial x} \equiv g(x, y)
\]
とすると, $g(x, y)$も微分可能で, 
\begin{equation}\label{eq-10}
  \frac{\partial f(x, y + \Delta y)}{\Delta x} 
  = g(x, y + \Delta y) 
  = g(x, y + \Delta y) - g(x, y) + g(x, y)
  = \frac{\partial g(x, y)}{\partial y}\Delta y + o(\Delta y) + g(x, y)
\end{equation}
式(\ref{eq-9})に代入すると, 
\begin{equation}\label{eq-11}
  \Delta f
  = \frac{\partial}{\partial y}\frac{\partial f(x, y)}{\partial x}\Delta x \Delta y
  + o(\Delta y)\Delta x
  + \frac{\partial f(x, y)}{\partial x}\Delta x 
  + o(\Delta x)
  + \frac{\partial f(x, y)}{\partial y}\Delta y
  + o(\Delta y)
\end{equation}

この議論は, 3変数の場合でも同様. 
\section{パラメータによる拘束}
\subsection{パラメータ1つの場合}
Sec.\ref{sec-1}の議論は, $f(x, y)$が定義されるすべての点集合$(x, y)$に対しての議論であったが, ここに, $x = x(t)$, $y = y(t)$という媒介変数を入れることは, すなわち, $(x, y)$で表される平面領域の中を通る, 連続曲線という, 部分集合に議論を限定するということであり, したがって, この媒介変数が存在しても, Sec.\ref{sec-1}の結論は有効である. 
ただし, 平面領域, その中を通る連続曲線, というのは, あくまで対象関数が連続, かつ微分可能という前提があるからである. 

式(\ref{eq-11})を利用すると, 
\begin{equation}\label{eq-12}
  \frac{d f}{d t}
  = \lim_{\Delta t \to 0}\frac{\Delta f}{\Delta t}
  = \lim_{\Delta t \to 0}\frac{\partial}{\partial y}\frac{\partial f(x, y)}{\partial x}\frac{\Delta x \Delta y}{\Delta t}
  + o(\Delta y)\frac{\Delta x}{\Delta t}
  + \frac{\partial f(x, y)}{\partial x}\frac{\Delta x}{\Delta t} 
  + \frac{o(\Delta x)}{\Delta t}
  + \frac{\partial f(x, y)}{\partial y}\frac{\Delta y}{\Delta t}
  + \frac{o(\Delta y)}{\Delta t}
\end{equation}

$x(t)$, $y(t)$ともに微分可能, すなわち, $\Delta x / \Delta t$, $\Delta y / \Delta t$ともに有限値に収束するということであり, それを利用すると, $\Delta t \to 0$において, 
式(\ref{eq-12})の第一項は, 
\[
  \frac{\Delta x \Delta y}{\Delta t} = \frac{\Delta x}{\Delta t} \frac{\Delta y}{\Delta t} \Delta t = O(\Delta t)
\]
により, 無くなる. 
第二項については, $\Delta y \to 0$なので, 無くなる. 
第四, 六項については, 
\[
  \frac{o(\Delta x)}{\Delta t} = \frac{o(\Delta x)}{\Delta x}\frac{\Delta x}{\Delta t}
\]
により, スモールオーの定義から, 無くなる. 
よって, 式(\ref{eq-12})は, 
\begin{equation}
  \frac{d f}{d t} = \frac{\partial f(x, y)}{\partial x}\frac{d x}{d t} + \frac{\partial f(x, y)}{\partial y}\frac{d y}{d t}
\end{equation}
となる. 

\subsection{2変数の場合}
パラメータが, 複数の場合は, 式(\ref{eq-11})における, $\Delta x$, $\Delta y$が, 複数パラメータにより変動する, という考え方になる. 
式(\ref{eq-3})の場合, パラメータ$u$,$v$ともに動きうるが, $u$による偏微分なので, $u$のみの変動により引き起こされる変化である. 式で書くならば, 
\begin{subequations}
  \begin{eqnarray}
    \Delta x = x(u + \Delta u, v) - x(u, v) \label{eq-14a}\\
    \Delta y = y(u + \Delta u, v) - y(u, v) \label{eq-14b}
  \end{eqnarray}
\end{subequations}
式(\ref{eq-14a}), (\ref{eq-14b})は, $x$, $y$が微分可能ならば, 
\begin{subequations}
  \begin{eqnarray}
    \Delta x = \frac{\partial x(u, v)}{\partial u} \Delta u + o(\Delta u)  \label{eq-15a} \\
    \Delta y = \frac{\partial y(u, v)}{\partial u} \Delta u + o(\Delta u)  \label{eq-15b}
  \end{eqnarray}
\end{subequations}
であり, 計算処理としては, 式(\ref{eq-12})と全く同様であり,
\[
  \frac{\Delta x \Delta y}{\Delta u}
  = \frac{\Delta x}{\Delta u}\frac{\Delta y}{\Delta u} \Delta u
  = \left(\frac{\partial x}{\partial u} + \frac{o(\Delta u)}{\Delta u}\right)
  \left(\frac{\partial y}{\partial u} + \frac{o(\Delta u)}{\Delta u}\right) \Delta u
\]
により, $\Delta u \to 0$において無くなり, 
第四, 六項の処理は全く同様なので無くなる. 
第三, 五項の処理は, 式(\ref{eq-15a}), (\ref{eq-15b})により, 
\[
  \frac{\Delta x}{\Delta u} \to \frac{\partial x}{\partial u}
\]
であるため, 式(\ref{eq-3})が成立する. 

\section{図形的な解釈}
$z = f(x, y)$とし, 点$(x, y, z)$の集合を考えると, これは, 3次元空間内での曲面を表す. 
$\Delta f = f(x + \Delta x, y + \Delta y) - f(x, y)$は, 曲面上の点$(x, y, z = f(x, y))$と, $(x + \Delta x, y + \Delta y, z^\prime = f(x + \Delta x, y + \Delta y))$の, $z$成分の差である. 

一方, $x$に関する偏微分の定義の分子, 
\[
  f(x + \Delta x, y) - f(x, y)
\]
は, 点$(x, y, z)$を通り, $y$が一定, すなわち, $xz$平面に並行な面で, 曲面を切断したときの, 曲線の, 点$(x, z)$における接線に沿った, $x$成分が$\Delta x$のベクトルの, $z$成分である. 
$y$に関する偏微分も同様に, $yz$平面に平行な面に見える, 曲線の, 点$(y, z)$における接線に沿った, $y$成分が$\Delta y$のベクトルの, $z$成分である. 
この事実により, 式(\ref{eq-11})が意味するところは, 
\begin{equation}
  \begin{pmatrix}
    \Delta x \\
    \Delta y \\
    f(x + \Delta x, y + \Delta y) - f(x, y)
  \end{pmatrix}
  =
  \begin{pmatrix}
    \Delta x \\
    0 \\
    f(x + \Delta x, y) - f(x, y)
  \end{pmatrix}
  +
  \begin{pmatrix}
    0 \\
    \Delta y \\
    f(x, y + \Delta y) - f(x, y)
  \end{pmatrix}
  +
  O(\Delta x) + O(\Delta y)
\end{equation}
右辺第一項, 第二項は, 切断面が, $xz$平面か, $yz$平面かの違いだけで, どちらも$(x, y, z)$における接線であること, 及び, それらが直行することから, 2本の独立したベクトルであり, 平面を構成する. 
式の意味はすなわち, 曲面$z = f(x, y)$上での, ある点$(x, y, z)$からの, 微小な近傍点$(x + \Delta x, y + \Delta y, f(x + \Delta x, y + \Delta y))$は, 
$(x, y, z)$における$x$方向, $y$方向の接線で張られる, 平面上の点$(x + \Delta x, y + \Delta y, z + \Delta z)$に対し, $\Delta x, \Delta y \to 0$の極限において, 一致する, という意味である. 
つまり, これは, この平面が, $z = f(x, y)$の$(x, y, z)$における接平面となる, ということを意味する. 
以上により, 偏微分の概念により, 一変数関数の微分の拡張となっていることがわかる. 

複数パラメータによる微分でも偏微分で表現できたという事実により, 微分可能な関数による座標変換であれば, 好きな座標系を選んで, 微分を計算することできる, ということになる. 
\end{document}
