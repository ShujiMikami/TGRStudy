\documentclass{jsarticle} \usepackage[dvipdfmx]{graphicx} \usepackage[dvipdfmx]{hyperref}
\usepackage{amsmath}
\usepackage{color}
\usepackage{colortbl}
\usepackage{arydshln}
\usepackage{mathtools}

\newcommand*{\mbold}[1]{\mbox{\boldmath $#1$}}

%\renewcommand*{\labelenumi}{(\arabic{enumi})}

\newcommand*{\transp}[1]{\prescript{t\!}{}{#1}}

\newcommand*{\grad}{{\rm grad}}
\newcommand*{\divg}{{\rm div}}
\newcommand*{\rot}{{\rm rot}}
\newcommand*{\trace}[1]{{\rm tr}\!{#1}}


\title{Theorem1.37}

\begin{document}
\maketitle

\begin{abstract}
  波動方程式
  
  \begin{equation}
    \left( \frac{\partial^2}{\partial x^2} - \frac{1}{c}\frac{\partial^2}{\partial t^2} \right) \phi(x, t) = 0
  \end{equation}
  の解 $\phi(x, t)$は, 
  \begin{equation}
    \phi(x, t) = f(x + ct) + g(x - ct)
  \end{equation}
\end{abstract}

\section{証明}
教科書通りにやればよい. 

$X = x + ct$, $Y = x - ct$としたとき, 

\begin{equation}
  \frac{\partial \phi}{\partial x}
    = \frac{\partial \phi}{\partial X}\frac{\partial X}{\partial x}
    + \frac{\partial \phi}{\partial Y}\frac{\partial Y}{\partial x}
    = \frac{\partial \phi}{\partial X} + \frac{\partial \phi}{\partial Y}
\end{equation}

\begin{equation}
  \frac{\partial \phi}{\partial t}
    = \frac{\partial \phi}{\partial X}\frac{\partial X}{\partial t}
    + \frac{\partial \phi}{\partial Y}\frac{\partial Y}{\partial t}
    = c \left( \frac{\partial \phi}{\partial X} - \frac{\partial \phi}{\partial Y} \right)
\end{equation}
なので, もう一度微分すると, 

\begin{equation}
  \frac{\partial^2 \phi}{\partial x^2} = \frac{\partial^2 \phi}{\partial x \partial X} + \frac{\partial^2 \phi}{\partial x \partial Y}
  = \frac{\partial^2 \phi}{\partial X^2} + \frac{\partial^2 \phi}{\partial Y^2} + 2\frac{\partial^2 \phi}{\partial X \partial Y}
\end{equation}

\begin{equation}
  \frac{\partial^2 \phi}{\partial t^2} = c \left( \frac{\partial^2 \phi}{\partial t \partial X} - \frac{\partial^2 \phi}{\partial t \partial Y} \right)
  = c^2 \left( \frac{\partial^2 \phi}{\partial X^2} - \frac{\partial^2 \phi}{\partial X \partial Y} \right) - c^2 \left( \frac{\partial^2 \phi}{\partial X \partial Y} - \frac{\partial^2 \phi}{\partial Y^2} \right)
\end{equation}

これらを代入すると, 

\begin{equation}
  \frac{\partial^2 \phi}{\partial X \partial Y} = 0
\end{equation}

まずは, 両辺を$X$で不定積分する. 

\begin{equation}
  \int \frac{\partial^2 \phi}{\partial X \partial Y} dX = \frac{\partial \phi}{\partial Y} = g(Y)
\end{equation}

$X$の偏微分で$0$になる, という条件は, $Y$のみの関数である, ということのため. 

これを, さらに$Y$で不定積分する. 

\begin{equation}
  \int \frac{\partial \phi}{\partial Y} dY = \phi = \int g(Y) dY + f(X)
\end{equation}

大一項を改めて, $Y$の関数$g(Y)$と呼べば, 

\begin{equation}
  \phi(X, Y) = g(Y) + f(X)
\end{equation}

であり, これは当初の式の形を満たす. 

十分条件の証明は, 代入すれば容易. 

\end{document}
