\documentclass{jsarticle} \usepackage[dvipdfmx]{graphicx} \usepackage[dvipdfmx]{hyperref}
\usepackage{amsmath}
\usepackage{color}
\usepackage{colortbl}
\usepackage{arydshln}
\usepackage{mathtools}

\newcommand*{\mbold}[1]{\mbox{\boldmath $#1$}}

%\renewcommand*{\labelenumi}{(\arabic{enumi})}

\newcommand*{\transp}[1]{\prescript{t\!}{}{#1}}

\newcommand*{\grad}{{\rm grad}}
\newcommand*{\divg}{{\rm div}}
\newcommand*{\rot}{{\rm rot}}
\newcommand*{\trace}[1]{{\rm tr}\!{#1}}


\title{Question1.25}

\begin{document}
\maketitle

\begin{abstract}
  $f(x, y) = x^2 y$を, 以下の経路$C_1$, $C_2$に沿って線積分する. 
  \begin{enumerate}
    \item $C_1:(t, 2-t)~~(0\leq t \leq 2)$
    \item $C_2:(2\sin\theta, 2\cos\theta)~~(0\leq \theta \leq \frac{\pi}{2})$
  \end{enumerate}

\end{abstract}

\section*{実行}
シンプルに実行するが, 念のため, $C_1$, $C_2$の媒介変数が全単射か確認する. 
$C_1$について, $x = t$なので, 全単射は自明. $y = 2 - t$も,オフセットされた直線なので自明.
よって, 連続全単射なので, 積分区間を区切る必要はない.
\begin{eqnarray}
  && \int_{C_1}f(x, y)ds \nonumber \\
  && = \int_0^2 t^2(2-t)\sqrt{1 + 1}dt \nonumber \\
  && = \sqrt{2} \left[\frac{2}{3}t^3 - \frac{t^4}{4} \right]^2_0 \nonumber \\
  && = \frac{4}{3}\sqrt{2}
\end{eqnarray}
\end{document}
