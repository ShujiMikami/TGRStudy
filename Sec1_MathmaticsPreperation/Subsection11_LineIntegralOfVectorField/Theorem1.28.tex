\documentclass{jsarticle} \usepackage[dvipdfmx]{graphicx} \usepackage[dvipdfmx]{hyperref}
\usepackage{amsmath}
\usepackage{color}
\usepackage{colortbl}
\usepackage{arydshln}
\usepackage{mathtools}

\newcommand*{\mbold}[1]{\mbox{\boldmath $#1$}}

%\renewcommand*{\labelenumi}{(\arabic{enumi})}

\newcommand*{\transp}[1]{\prescript{t\!}{}{#1}}

\newcommand*{\grad}{{\rm grad}}
\newcommand*{\divg}{{\rm div}}
\newcommand*{\rot}{{\rm rot}}
\newcommand*{\trace}[1]{{\rm tr}\!{#1}}


\title{Theorem1.28}

\begin{document}
\maketitle

\begin{abstract}
  ベクトル場$\mbold{F}(\mbold{x})$に対し, $\mbold{F} = \grad{g(\mbold{x})}$が成立する, スカラー場$g(\mbold{x})$が存在するとき, 
  2定点$A$, $B$, を結ぶ曲線$C$に沿った線積分は, $C$の取り方によらず, 一定値になる.  
\end{abstract}

\section*{証明}
$C$が$t$でパラメータ表示されているとすると, 
\begin{eqnarray}
  && \int_C \mbold{F}(\mbold{x}) \cdot d\mbold{s} = \int_a^b \mbold{F}(\mbold{x}(t)) \cdot \left(\frac{d\mbold{x}}{dt}dt \right) \nonumber\\
  && = \int_a^b (\nabla g(\mbold{x}(t)) \cdot \frac{d \mbold{x}}{dt}dt \nonumber\\
  && = \int_a^b \sum_i \left( \frac{\partial g}{\partial x_i}\frac{d x_i}{dt} \right)dt \nonumber\\
  && = \int_a^b \frac{d g}{dt} dt = g(b) - g(a)
\end{eqnarray}
となり, $\mbold{x}(t)$の関数形に依存しない結果となる. 
\end{document}
