\documentclass{jsarticle} \usepackage[dvipdfmx]{graphicx} \usepackage[dvipdfmx]{hyperref}
\usepackage{amsmath}
\usepackage{color}
\usepackage{colortbl}
\usepackage{arydshln}
\usepackage{mathtools}

\newcommand*{\mbold}[1]{\mbox{\boldmath $#1$}}

%\renewcommand*{\labelenumi}{(\arabic{enumi})}

\newcommand*{\transp}[1]{\prescript{t\!}{}{#1}}

\newcommand*{\grad}{{\rm grad}}
\newcommand*{\divg}{{\rm div}}
\newcommand*{\rot}{{\rm rot}}
\newcommand*{\trace}[1]{{\rm tr}\!{#1}}


\title{Question1.26}

\begin{document}
\maketitle

\begin{abstract}
  ベクトル場$\mbold{F}(x, y) = (x + 1, xy)$に対し, 曲線$C:(t^2, t)$ $(0\leq t \leq 2)$上を, 線積分する. 
\end{abstract}

\section*{実行}
$x = t^2$, $y = t$は, $t$の定義域の中では, 全単射. よって単純に計算するだけ. 

\begin{eqnarray}
  && \int_C \mbold{F} d\mbold{s} = \int_0^2 ((t^2 + 1)\cdot 2t + t^3)dt \nonumber \\
  && = \left[ \frac{3 t^4}{4} + t^2 \right]_0^2 = 16
\end{eqnarray}

教科書では, $x$, $y$の直接計算も試しているが, 省略. 

\end{document}
