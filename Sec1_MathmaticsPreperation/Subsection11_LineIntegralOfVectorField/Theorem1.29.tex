\documentclass{jsarticle} \usepackage[dvipdfmx]{graphicx} \usepackage[dvipdfmx]{hyperref}
\usepackage{amsmath}
\usepackage{color}
\usepackage{colortbl}
\usepackage{arydshln}
\usepackage{mathtools}

\newcommand*{\mbold}[1]{\mbox{\boldmath $#1$}}

%\renewcommand*{\labelenumi}{(\arabic{enumi})}

\newcommand*{\transp}[1]{\prescript{t\!}{}{#1}}

\newcommand*{\grad}{{\rm grad}}
\newcommand*{\divg}{{\rm div}}
\newcommand*{\rot}{{\rm rot}}
\newcommand*{\trace}[1]{{\rm tr}\!{#1}}


\title{Theorem1.29}

\begin{document}
\maketitle

\begin{abstract}
  ベクトル場$\mbold{F}(\mbold{x})$に対し, $\rot\mbold{F} = \mbold{0}$が成立するとき, 
  2定点$A$, $B$, を結ぶ曲線$C$に沿った$\mbold{F}$の線積分は, $C$の取り方によらず, 一定値になる.  
\end{abstract}

\section*{証明}
Theorem1.23より, $\rot$がゼロになることと, スカラーポテンシャルの存在は, 必要十分であった. したがって, 
$\mbold{F} = \nabla g$なる, スカラーポテンシャル$g$が存在するので, 本Theoremは, Theorem1.29と同値である. 
\end{document}
