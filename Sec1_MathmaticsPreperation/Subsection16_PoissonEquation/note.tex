\documentclass{jsarticle} \usepackage[dvipdfmx]{graphicx} \usepackage[dvipdfmx]{hyperref}
\usepackage{amsmath}
\usepackage{color}
\usepackage{colortbl}
\usepackage{arydshln}
\usepackage{mathtools}

\newcommand*{\mbold}[1]{\mbox{\boldmath $#1$}}

%\renewcommand*{\labelenumi}{(\arabic{enumi})}

\newcommand*{\transp}[1]{\prescript{t\!}{}{#1}}

\newcommand*{\grad}{{\rm grad}}
\newcommand*{\divg}{{\rm div}}
\newcommand*{\rot}{{\rm rot}}
\newcommand*{\trace}[1]{{\rm tr}\!{#1}}


\title{note}

\begin{document}
\maketitle

\begin{abstract}
  速度$c$で一様物質中を伝播する1次元の波動の量$\phi(x, t)$は, 以下の方程式に従う. 

  \begin{equation}
    \left( \frac{\partial^2}{\partial x^2} - \frac{1}{c}\frac{\partial^2}{\partial t^2} \right) \phi(x, t) = 0
  \end{equation}

\end{abstract}

1次元の物質の内部を, 大量の小さなバネで接続された質量$m$の質点の集合であると見立て, 
質点の変位を$\phi_i(t)$とし, 運動方程式を立てる. 

\begin{equation}
  m \frac{\partial^2 \phi_i}{\partial t^2} = -k(\phi_i - \phi_{i + 1}) - k(\phi_i - \phi_{i - 1})
\end{equation}

バネは, 同じものを直列に接続すると, バネ定数が$1/2$に見える. すなわち, 
同一のバネを直列接続した場合, バネ一つあたりの縮み量が半減し, その結果, バネが発する弾性力も半分になる. 
つまり, システム全体としては, 同じ押し込み量に対し, 力が半減しているので, バネ定数が半減しているように見える. 
これを連続長にすると, バネ定数は, 長さに反比例する. 

よって, 有限長の1次元の物質の弾性定数が, ある有限値を取るならば, ミクロに分割したバネの定数は, その分割数に比例して大きくなる. 
すなわち, ある有限長$L$の物質を, $N$分割したモデルでは, 

\begin{equation}
  m \frac{\partial^2 \phi_i}{\partial t^2} = -kN(\phi_i - \phi_{i + 1}) - kN(\phi_i - \phi_{i - 1})
\end{equation}

また, 質点の質量も, 一次元物質の質量が有限の$M$であったなら, その$N$分割をしているので, 

\begin{equation}
  \frac{M}{N} \frac{\partial^2 \phi_i}{\partial t^2} = -kN(\phi_i - \phi_{i + 1}) - kN(\phi_i - \phi_{i - 1})
\end{equation}

$\Delta x = L/N$とすると, 

\begin{equation}
  \frac{M}{L} \Delta x \frac{\partial^2 \phi_i}{\partial t^2} = -\frac{1}{\Delta x} \left( kL(\phi_i - \phi_{i + 1}) - kL(\phi_i - \phi_{i - 1}) \right)
\end{equation}

密度, すなわち$\rho = M/L$を使うと, 

\begin{equation}
  \rho \frac{\partial^2 \phi_i}{\partial t^2} = -\frac{kL}{\Delta x^2} ( 2 \phi_i - \phi_{i + 1} - \phi_{i - 1}) )
\end{equation}

ここで, $\phi$を, $i$の代わりに, あるポジション$x$での変位$\phi(x, t)$とすると(ただし, この段階ではまだ$x$は離散的の認識), 

\begin{equation}
  \rho \frac{\partial^2 \phi(x, t)}{\partial t^2} = -\frac{kL}{\Delta x^2} ( 2 \phi(x, t) - \phi(x + \Delta x, t) - \phi(x - \Delta x, t) )
\end{equation}

$\phi$を, $x$について, 関数としては連続であるとみなせるほど, 隣同士の距離が十分小さいと考え, テイラー展開を2次まで利用すると, 

\begin{eqnarray}
  \rho \frac{\partial^2 \phi(x, t)}{\partial t^2}
  = && -\frac{kL}{\Delta x^2} ( \nonumber \\
    && 2 \phi(x, t) \nonumber \\ 
    && - \phi(x, t) - \frac{\partial \phi}{\partial x}(x, t) \Delta x - \frac{1}{2!}\frac{\partial^2 \phi}{\partial x^2}(x, t)\Delta x^2 \nonumber\\
    && - \phi(x, t) + \frac{\partial \phi}{\partial x}(x, t) \Delta x - \frac{1}{2!}\frac{\partial^2 \phi}{\partial x^2}(x, t)\Delta x^2 \nonumber\\
        ) \nonumber \\
    = kL\frac{\partial^2 \phi}{\partial x^2}(x, t) 
\end{eqnarray}

これは, 当初の形になっている. 


\end{document}
