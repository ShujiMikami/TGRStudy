\documentclass{jsarticle} \usepackage[dvipdfmx]{graphicx} \usepackage[dvipdfmx]{hyperref}
\usepackage{amsmath}
\usepackage{color}
\usepackage{colortbl}
\usepackage{amssymb}
\usepackage{arydshln}
\usepackage{mathtools}

\newcommand*{\mbold}[1]{\mbox{\boldmath $#1$}}

%\renewcommand*{\labelenumi}{(\arabic{enumi})}

\newcommand*{\transp}[1]{\prescript{t\!}{}{#1}}

\newcommand*{\grad}{{\rm grad}}
\newcommand*{\divg}{{\rm div}}
\newcommand*{\rot}{{\rm rot}}
\newcommand*{\trace}[1]{{\rm tr}\!{#1}}


\title{note}

\begin{document}
\maketitle

\begin{abstract}
  領域$V \in \mathbb{R}^3$で定義され, その外で$0$になる関数$f(\mbold{x})$に対し, 
  
  \begin{equation}
    \phi(\mbold{x}) = \frac{1}{4\pi} \oint_V \frac{f(\mbold{y})}{|\mbold{x} - \mbold{y}|}d\mbold{y}
  \end{equation}
  のは, 
  \begin{equation}
    \varDelta \phi(\mbold{x}) = -f(\mbold{x})
  \end{equation}
\end{abstract}

を満たす. 

\section{lemma1}
Gaussの発散定理, 及び, Greenの定理は, 単純な閉領域から, 部分領域を除いた, いわゆる中空構造の空間に対しても成立する. 

すなわち, 中空構造の空間を, 内部の面と, 外部の面を貫くような, 細い直方体で分割する. 
この一つの直方体は, 単純な閉領域であり, Gaussの発散定理や, Greenの定理が成立することは自明である. 
そして, 隣り合う直方体同士は, ある一つの面を共有しており, この面のそれぞれの直方体領域にとっての法線ベクトルは, 互いに反対を向いているので, 
その面での面積分は互いに打ち消しあう. 
よって, これを全直方体領域の和で考えると, 結局, 中空部には, 内向きにベクトルを取ったうえで, 外側の面の積分と, 中空部の面の積分を足した結果が, 中空な領域の体積積分と等しくなる. 
体積積分が, 中空でない空間から, 中空の内部空間での積分を引いたものである, という考えても, 同様の結論にたどり着くことができる. 

\section{証明}


\end{document}
