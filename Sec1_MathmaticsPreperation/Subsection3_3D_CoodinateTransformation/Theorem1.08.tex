\documentclass{jsarticle}
\usepackage[dvipdfmx]{graphicx}
\usepackage{amsmath}
\usepackage{color}
\usepackage{colortbl}
\usepackage{arydshln}

\newcommand*{\mbold}[1]{\mbox{\boldmath $#1$}}

\renewcommand*{\labelenumi}{(\arabic{enumi})}

\title{Theorem1.08}

\begin{document}
\maketitle

\begin{abstract}
  直交座標系$\Sigma_{o}$, $\Sigma^\prime_{o}$は, 原点と, スケールが一致しているとき, 
  ベクトル$\mbold{p}$が, 
  $\Sigma_{o}$において, 
  \[
    \mbold{p}
    =
    \begin{pmatrix}
      a_1 \\
      a_2 \\
      a_3
    \end{pmatrix}
  \]
  $\Sigma^\prime_{o}$において, 
  \[
    \mbold{p}
    =
    \begin{pmatrix}
      p^\prime_1 \\
      p^\prime_2 \\
      p^\prime_3
    \end{pmatrix}
  \]
  と成分表示されるとき, 
  \begin{equation}
    {}^\exists U, 
    \begin{pmatrix}
      p^\prime_1 \\
      p^\prime_2 \\
      p^\prime_3
    \end{pmatrix}
    = U
    \begin{pmatrix}
      p_1 \\
      p_2 \\
      p_3
    \end{pmatrix}
    ,
    U{}^t U = E
  \end{equation}
\end{abstract}

\section*{本証明}
$\Sigma_{o}$の基底ベクトルを, $\{ \mbold{e}_i \}_{i = 1}^{3}$とすると, 
\begin{equation}
  \mbold{p} = \sum_{i = 1}^3 p_i \mbold{e}_i
\end{equation}

$\Sigma^\prime_{o}$の基底ベクトルを, $\{ \mbold{e}^\prime_i \}_{i = 1}^{3}$とすると, 
$\Sigma^\prime_{o}$での$1$方向成分は, 
\begin{equation}
  p^\prime_1 = \mbold{e}^\prime_1 \cdot \mbold{p}
  =\sum_{i = 1}^3 p_i \mbold{e}^\prime_1 \cdot \mbold{e}_i
\end{equation}
他の成分も同様なので, 縦に並べると, 
\begin{equation}
  \begin{pmatrix}
    p^\prime_1 \\
    p^\prime_2 \\
    p^\prime_3
  \end{pmatrix}
  =
  \begin{pmatrix}
    p_1 \mbold{e}^\prime_1 \cdot \mbold{e}_1 + p_2 \mbold{e}^\prime_1 \cdot \mbold{e}_2 + p_3 \mbold{e}^\prime_1 \cdot \mbold{e}_3 \\
    p_1 \mbold{e}^\prime_2 \cdot \mbold{e}_1 + p_2 \mbold{e}^\prime_2 \cdot \mbold{e}_2 + p_3 \mbold{e}^\prime_2 \cdot \mbold{e}_3 \\
    p_1 \mbold{e}^\prime_3 \cdot \mbold{e}_1 + p_2 \mbold{e}^\prime_3 \cdot \mbold{e}_2 + p_3 \mbold{e}^\prime_3 \cdot \mbold{e}_3 
  \end{pmatrix}
  =
  \begin{pmatrix}
    \mbold{e}^\prime_1 \cdot \mbold{e}_1 & \mbold{e}^\prime_1 \cdot \mbold{e}_2 & \mbold{e}^\prime_1 \cdot \mbold{e}_3 \\
    \mbold{e}^\prime_2 \cdot \mbold{e}_1 & \mbold{e}^\prime_2 \cdot \mbold{e}_2 & \mbold{e}^\prime_2 \cdot \mbold{e}_3 \\
    \mbold{e}^\prime_3 \cdot \mbold{e}_1 & \mbold{e}^\prime_3 \cdot \mbold{e}_2 & \mbold{e}^\prime_3 \cdot \mbold{e}_3
  \end{pmatrix}
  \begin{pmatrix}
    p_1 \\
    p_2 \\
    p_3
  \end{pmatrix}
\end{equation}
よって, 
\begin{equation}
  U = 
  \begin{pmatrix}
    \mbold{e}^\prime_1 \cdot \mbold{e}_1 & \mbold{e}^\prime_1 \cdot \mbold{e}_2 & \mbold{e}^\prime_1 \cdot \mbold{e}_3 \\
    \mbold{e}^\prime_2 \cdot \mbold{e}_1 & \mbold{e}^\prime_2 \cdot \mbold{e}_2 & \mbold{e}^\prime_2 \cdot \mbold{e}_3 \\
    \mbold{e}^\prime_3 \cdot \mbold{e}_1 & \mbold{e}^\prime_3 \cdot \mbold{e}_2 & \mbold{e}^\prime_3 \cdot \mbold{e}_3
  \end{pmatrix}
\end{equation}
により, 座標変換が与えられることがわかった. この行列$U$の$(i, j)$成分は, 
\begin{equation}
  U_{ij} = \mbold{e}^\prime_i \cdot \mbold{e}_j
\end{equation}
と表されることに注意. 
$(i, j)$成分表示を利用して, ${}^t U U$を計算すると, 
\begin{equation}\label{eq-7}
  ({}^t U U)_{ij} = \sum_{k = 1}^3 ({}^t U)_{ik} U_{kj}
  = \sum_{k = 1}^3 U_{ki}U_{kj}
\end{equation}
ここで, $U_{ki} = \mbold{e}^\prime_k \mbold{e}_i$は, $\mbold{e}_i$の, $\Sigma^\prime_{o}$系での, $k$成分であるので, それを全3成分に対して加算している, 
式(\ref{eq-7})の右辺は, $\mbold{e}_i$と, $\mbold{e}_j$の内積である. 
$\Sigma_{o}$が直交座標系であったことから, 
\begin{subequations}
  \begin{eqnarray}
    \mbold{e}_i \cdot \mbold{e}_j = 0, i \neq j \\
    \mbold{e}_i \cdot \mbold{e}_j = 1, i = j
  \end{eqnarray}
\end{subequations}
なので, ${}^t U U$は, 対角成分が$1$, それ以外のすべての成分が$0$の正方行列, すなわち単位行列$E$である. 
また, $U {}^t U$については, 式(\ref{eq-7})の$kj$成分側を反転させるだけで, 議論に変化はなく, したがって, 
\[
  {}^t U U = U {}^t U = E
\]
である. 

\end{document}
