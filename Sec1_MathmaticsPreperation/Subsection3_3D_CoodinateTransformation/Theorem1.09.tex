\documentclass{jsarticle}
\usepackage[dvipdfmx]{graphicx}
\usepackage{amsmath}
\usepackage{color}
\usepackage{colortbl}
\usepackage{arydshln}
\usepackage{mathtools}

\newcommand*{\mbold}[1]{\mbox{\boldmath $#1$}}

\renewcommand*{\labelenumi}{(\arabic{enumi})}

\newcommand*{\transp}[1]{\prescript{t\!}{}{#1}}

\title{Theorem1.09}

\begin{document}
\maketitle

\begin{abstract}
  直交座標系$\Sigma_{o} \to \Sigma^\prime_{o}$は, 直交変換.
  位置ベクトル$\mbold{a}$, $\mbold{b}$が, 
  $\Sigma_{o}$において, 
  \begin{eqnarray*}
    \mbold{a}
    =
    \begin{pmatrix}
      a_1 \\
      a_2 \\
      a_3
    \end{pmatrix} \\
    \mbold{b}
    =
    \begin{pmatrix}
      b_1 \\
      b_2 \\
      b_3
    \end{pmatrix} \\
  \end{eqnarray*}
  $\Sigma^\prime_{o}$において, 
  \begin{eqnarray*}
    \mbold{a}
    =
    \begin{pmatrix}
      a^\prime_1 \\
      a^\prime_2 \\
      a^\prime_3
    \end{pmatrix} \\
    \mbold{b}
    =
    \begin{pmatrix}
      b^\prime_1 \\
      b^\prime_2 \\
      b^\prime_3
    \end{pmatrix} \\
  \end{eqnarray*}

  と成分表示されるとき, 
  \begin{equation}
    a_1 b_1 + a_2 b_2 + a_3 b_3 = a^\prime_1 b^\prime_1 + a^\prime_2 b^\prime_2 + a^\prime_3 b^\prime_3
  \end{equation}
\end{abstract}

\section*{本証明}
Theorem1.08により, 直交行列$U$があって, 
\begin{equation}
  \begin{pmatrix}
    a^\prime_1 \\
    a^\prime_2 \\
    a^\prime_3
  \end{pmatrix}
  =
  U
  \begin{pmatrix}
    a_1 \\
    a_2 \\
    a_3
  \end{pmatrix}
\end{equation}

\begin{equation}
  a^\prime_1 b^\prime_1 + a^\prime_2 b^\prime_2 + a^\prime_3 b^\prime_3
  = 
  \begin{pmatrix}
    a^\prime_1 & a^\prime_2 & a^\prime_3 
  \end{pmatrix}
  \begin{pmatrix}
    a^\prime_1 \\
    a^\prime_2 \\
    a^\prime_3
  \end{pmatrix}
  = \transp{ 
      \begin{pmatrix}
        a^\prime_1 \\
        a^\prime_2 \\
        a^\prime_3
      \end{pmatrix}
    }
  \begin{pmatrix}
    a^\prime_1 \\
    a^\prime_2 \\
    a^\prime_3
  \end{pmatrix}
\end{equation}
なので, $U$を使用して書き換えると, 
\begin{equation}
  a^\prime_1 b^\prime_1 + a^\prime_2 b^\prime_2 + a^\prime_3 b^\prime_3
  = \transp{ 
      \begin{pmatrix}
        a^\prime_1 \\
        a^\prime_2 \\
        a^\prime_3
      \end{pmatrix}
    }
  \begin{pmatrix}
    a^\prime_1 \\
    a^\prime_2 \\
    a^\prime_3
  \end{pmatrix}
  = \transp{
      \left(
      U
      \begin{pmatrix}
        a_1 \\
        a_2 \\
        a_3
      \end{pmatrix}
      \right)
    }
    U
  \begin{pmatrix}
    a_1 \\
    a_2 \\
    a_3
  \end{pmatrix}
  =
  \begin{pmatrix}
    a_1 & a_2 & a_3 \\
  \end{pmatrix}
  \transp{U}
  U
  \begin{pmatrix}
    a_1 \\
    a_2 \\
    a_3
  \end{pmatrix}
\end{equation}
Theorem1.08より, $\transp{U} U = E$なので, 

\begin{equation}
  a^\prime_1 b^\prime_1 + a^\prime_2 b^\prime_2 + a^\prime_3 b^\prime_3
  =
  \begin{pmatrix}
    a_1 & a_2 & a_3 \\
  \end{pmatrix}
  \begin{pmatrix}
    a_1 \\
    a_2 \\
    a_3
  \end{pmatrix}
  =
  a_1 b_1 + a_2 b_2 + a_3 b_3
\end{equation}

\end{document}
