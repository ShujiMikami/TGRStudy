\documentclass{jsarticle} 
\usepackage[dvipdfmx]{graphicx} 
\usepackage[dvipdfmx]{hyperref}
\usepackage{amsmath}
\usepackage{color}
\usepackage{amssymb}
\usepackage{colortbl}
\usepackage{arydshln}
\usepackage{mathtools}

\newcommand*{\mbold}[1]{\mbox{\boldmath $#1$}}

%\renewcommand*{\labelenumi}{(\arabic{enumi})}

\newcommand*{\transp}[1]{\prescript{t\!}{}{#1}}

\newcommand*{\grad}{{\rm grad}}
\newcommand*{\divg}{{\rm div}}
\newcommand*{\rot}{{\rm rot}}
\newcommand*{\trace}[1]{{\rm tr}\!{#1}}


\title{Formula1.36}

\begin{document}
\maketitle

\begin{abstract}
  $\mbold{x}, \mbold{a} \in \mathbb{R}^3$, $\mbold{a} = const$. $\mbold{x} \neq \mbold{a}$.
  閉曲面$S$とそれにより囲まれた閉領域$D$.
  \begin{equation*}
    \mbold{F}(\mbold{x}) = \frac{\mbold{x} - \mbold{a}}{|\mbold{x} - \mbold{a}|^3}
  \end{equation*}
  に対し, 
  
  \begin{equation}
    \int_S \mbold{F}\cdot\mbold{n}dS = 
    \begin{cases}
      4\pi & \mbold{a} \in D \\
      0    & \mbold{a} \notin D 
    \end{cases}
  \end{equation}


\end{abstract}

\section*{計算}
\subsection{$\mbold{a} \notin D$のとき}
$\mbold{a} \notin D$のときは簡単である. 
Gaussの発散定理, Theorem1.33が使える. つまり, 
\begin{equation}
  \int_S \mbold{F}\cdot\mbold{n}dS = \int_D \divg \mbold{F} dV = 0
\end{equation}

$\mbold{a} \in D$の場合は, 教科書では, 簡単に$4\pi$としているが, 
実は, $2\pi$の場合がある. 
以下の2パターンで考える. 

\subsection{$\mbold{a} \in D, \notin S$のとき}
$\mbold{a}$が, $D$の境界である, $S$上ではなく, 完全に, $S$の内側にある時である. 
教科書では, $S$を勝手に球に置き換えているが, これは, 結果として正しいが, 論理としては飛躍がある. 
まず, $S$は, 勝手に決めていいものではなく, 決められたものであるため, 任意の$S$で成立する論理を立てねばならない. 

Theorem1.33のlemma2を利用する. $D$に対し, 完全に$D$に内包されるように, 十分に小さな半径$r$の, $\mbold{a}$を中心とした球$D^\prime$を考え, 
$D^\prime$の表面を$S^\prime$とする. 
Theorem1.33のlemma2により, $S$での面積分は, $D^\prime$の表面での面積分と, $D$から, $D^\prime$をくりぬいて残った領域の面積分の和で表される. 
一方で, 後者は, 特性として$\mbold{a} \notin D$の時と全く同一ケースなので, $0$になる. 
前者は, 
\begin{equation}
  \int_{S^\prime} \mbold{F}\cdot \mbold{n}dS
  = \int_{S^\prime} \frac{\mbold{x} - \mbold{a}}{|\mbold{x} - \mbold{a}|^3} \cdot \frac{\mbold{x} - \mbold{a}}{|\mbold{x} - \mbold{a}|}dS 
  = \frac{1}{r^2}\int_{S^\prime} dS = \frac{4\pi r^2}{r^2} = 4\pi
\end{equation}

よって, $\mbold{a} \in D$かつ, $\mbold{a} \notin S$のときは, 積分値は$4\pi$.

\subsection{$\mbold{a} \in S$のとき}
このときは, 完全内包時と同様のロジックであるが, $D^\prime$を, $\mbold{a}$を中心とした, 半球にする. 
くりぬかれた側の積分が$0$になるのは全く同様. 
残った半球の積分について, 
半球の, 切り口である, $\mbold{a}$を中心とする円$S^\prime_1$での積分は, 
直交変換により, 原点が$\mbold{a}$, 円が$xy$平面上, 半球の内側に$z$座標が来るように座標を取れば, 
$z = 0$であることを考慮すると, 
\begin{equation}
  \int_{S^\prime} \mbold{F}\cdot \mbold{n}dS
  = \int_{S^\prime_1} \frac{\mbold{x} - \mbold{a}}{|\mbold{x} - \mbold{a}|^3} \cdot (0, 0, 1)dS 
  = \int_{S^\prime_1} \frac{z}{\sqrt{x^2 + y^2}^3} = 0
\end{equation}

半球面$S^\prime_2$での計算は, $\mbold{a} \notin S$の時の計算結果を利用すれば, その半分の$2\pi$である. 

以上により, 
\begin{equation}
  \int_S \mbold{F}\cdot\mbold{n}dS = 
  \begin{cases}
    4\pi & \mbold{a} \in D, \notin S \\
    2\pi & \mbold{a} \in D, \in S \\
    0    & \mbold{a} \notin D 
  \end{cases}
\end{equation}
である. 

\end{document}
