\documentclass{jsarticle}
\usepackage[dvipdfmx]{graphicx}
\usepackage[dvipdfmx]{hyperref}
\usepackage{amsmath}
\usepackage{color}
\usepackage{colortbl}
\usepackage{arydshln}
\usepackage{mathtools}

\newcommand*{\mbold}[1]{\mbox{\boldmath $#1$}}

%\renewcommand*{\labelenumi}{(\arabic{enumi})}

\newcommand*{\transp}[1]{\prescript{t\!}{}{#1}}

\newcommand*{\grad}{{\rm grad}}
\newcommand*{\divg}{{\rm div}}
\newcommand*{\rot}{{\rm rot}}
\newcommand*{\trace}[1]{{\rm tr}\!{#1}}


\title{Formula1.19}

\begin{document}
\maketitle

\begin{abstract}
  \begin{equation}
    \rot\rot\mbold{A}(\mbold{x}) = \grad \divg \mbold{A}(\mbold{x}) - \varDelta \mbold{A}(\mbold{x})
  \end{equation}
  ただし, $\varDelta \mbold{A}(\mbold{x}) = (\varDelta A_x(\mbold{x}), \varDelta A_y(\mbold{x}),\varDelta A_z(\mbold{x}))$
\end{abstract}

\section{lemma}
エディントンのエプシロンを, 以下のように定義する. 
\begin{equation}
  \epsilon_{ijk} = 
    \begin{cases}
      1 & (i, j, k) = (1, 2, 3), (2, 3, 1), (3, 1, 2) \\
      -1 & (i, j, k) = (1, 3, 2), (2, 1, 3), (3, 2, 1) \\
      0 & (otherwise)
    \end{cases}
\end{equation}
外積が, 
\begin{equation}
  (\mbold{a} \times \mbold{b})_i = \sum_{j, k}\epsilon_{ijk}a_j b_k
\end{equation}
$\rot$が, 
\begin{equation}
  (\rot\mbold{A}(\mbold{x}))_i = \sum_{j, k}\epsilon_{ijk}\frac{\partial A_k}{\partial x_j}
\end{equation}
となる. 
\subsection{証明}
成分ごとに計算する. 
\begin{eqnarray}
  \sum_{j, k}\epsilon_{1jk}a_j b_k = \epsilon_{123}a_2 b_3 + \epsilon_{132}a_3 b_2 = a_2 b_3 - a_3 b_2 = (\mbold{a} \times \mbold{b})_1 \nonumber \\
  \sum_{j, k}\epsilon_{2jk}a_j b_k = \epsilon_{231}a_3 b_1 + \epsilon_{213}a_1 b_3 = a_3 b_1 - a_1 b_3 = (\mbold{a} \times \mbold{b})_2 \nonumber \\
  \sum_{j, k}\epsilon_{3jk}a_j b_k = \epsilon_{312}a_1 b_2 + \epsilon_{321}a_2 b_1 = a_1 b_2 - a_2 b_1 = (\mbold{a} \times \mbold{b})_3
\end{eqnarray}
同様に, 
\begin{eqnarray}
  \sum_{j, k}\epsilon_{1jk}\frac{\partial A_k}{\partial x_j}
  = \epsilon_{123}\frac{\partial A_3}{\partial x_2} + \epsilon_{132}\frac{\partial A_2}{\partial x_3}
  = \frac{\partial A_3}{\partial x_2} - \frac{\partial A_2}{\partial x_3} = (\rot \mbold{A})_1 \nonumber \\
  \sum_{j, k}\epsilon_{2jk}\frac{\partial A_k}{\partial x_j}
  = \epsilon_{231}\frac{\partial A_1}{\partial x_3} + \epsilon_{213}\frac{\partial A_3}{\partial x_1}
  = \frac{\partial A_3}{\partial x_2} - \frac{\partial A_2}{\partial x_3} = (\rot \mbold{A})_2 \nonumber \\
  \sum_{j, k}\epsilon_{3jk}\frac{\partial A_k}{\partial x_j}
  = \epsilon_{312}\frac{\partial A_2}{\partial x_1} + \epsilon_{321}\frac{\partial A_1}{\partial x_2}
  = \frac{\partial A_2}{\partial x_1} - \frac{\partial A_1}{\partial x_2} = (\rot \mbold{A})_3 \nonumber \\
\end{eqnarray}
これはつまり, ナブラの各成分が, あたかもベクトルの成分であるかのように, 計算を行って良いことを示している. 
\section{lemma}
エディントンのエプシロンに対し, 
\begin{subequations}
  \begin{eqnarray}
    && \sum_{i, j, k}\epsilon_{ijk}^2 = 6 \\
    && \sum_{k, m}\epsilon_{ikm}\epsilon_{jkm} = 2\delta_{ij} \\
    && \sum_{i}\epsilon_{ijk}\epsilon_{ilm} = \delta_{jl}\delta_{km} - \delta_{jm}\delta_{kl}
  \end{eqnarray}
\end{subequations}
\subsection{証明}
非自明なパターンは, 6つのみで, 大きさが$1$なので, 1つ目は自明. 
2つ目の式は, 定義の中で, $\epsilon$が非自明な値を取る, $k$, $m$が共通する複数パターンがないことから, 
$i = j$の項しか残らない. 
また, $i = j$のときは, ある固定された$i$に対し, 非自明な値を取るパターンは, $1$と$-1$の2パターンしかないことから, 
加算した結果は$2$. 
よって, 式が成り立つ. 

\section*{本証明}
定義通り計算するだけ. 関数の引数は省略する. 
\begin{eqnarray}
  && (\rot\rot\mbold{A})_i \nonumber \\
  && = \rot \nonumber \\
  && = \sum_i (\frac{\partial f}{\partial x_i} A_i + f \frac{\partial A_i}{\partial x_i}) \nonumber \\
  && = \sum_i (\grad f)_i A_i + f \divg\mbold{A} \nonumber \\
  && = \grad f \cdot \mbold{A} + f \divg\mbold{A}
\end{eqnarray}


\end{document}
