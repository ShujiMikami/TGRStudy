\documentclass{jsarticle}
\usepackage[dvipdfmx]{graphicx}
\usepackage[dvipdfmx]{hyperref}
\usepackage{amsmath}
\usepackage{color}
\usepackage{colortbl}
\usepackage{arydshln}
\usepackage{mathtools}

\newcommand*{\mbold}[1]{\mbox{\boldmath $#1$}}

%\renewcommand*{\labelenumi}{(\arabic{enumi})}

\newcommand*{\transp}[1]{\prescript{t\!}{}{#1}}

\newcommand*{\grad}{{\rm grad}}
\newcommand*{\divg}{{\rm div}}
\newcommand*{\rot}{{\rm rot}}
\newcommand*{\trace}[1]{{\rm tr}\!{#1}}


\title{Formula1.22}

\begin{document}
\maketitle

\begin{abstract}
  \begin{equation}
    \divg\rot \mbold{A}(\mbold{x}) = 0
  \end{equation}
\end{abstract}

\section*{本証明}
Formula1.19のlemmaを利用し, 成分計算する. 引数は省略する. 

\begin{eqnarray}
  && \divg \rot \mbold{A} \nonumber \\
  && = \sum_i \frac{\partial}{\partial x_i} \sum_{j, k}\epsilon_{ijk}\frac{\partial A_k}{\partial x_j} \nonumber \\
  && = \frac{1}{2} \sum_k \left( \sum_{i, j} \epsilon_{ijk}\frac{\partial^2 A_k}{\partial x_i \partial x_j} + \sum_{i, j} \epsilon_{jik}\frac{\partial^2 A_k}{\partial x_j \partial x_i}\right)\nonumber \\
  && = \frac{1}{2} \sum_k \left( \sum_{i, j} \epsilon_{ijk}\frac{\partial^2 A_k}{\partial x_i \partial x_j} + \sum_{j, k} (-\epsilon_{ijk})\frac{\partial^2 A_k}{\partial x_j \partial x_i}\right) \nonumber \\
  && = 0
\end{eqnarray}
やっていることは, Formula1.21とほぼ同じ. 


\end{document}
