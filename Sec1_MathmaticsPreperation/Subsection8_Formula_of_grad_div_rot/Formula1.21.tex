\documentclass{jsarticle}
\usepackage[dvipdfmx]{graphicx}
\usepackage[dvipdfmx]{hyperref}
\usepackage{amsmath}
\usepackage{color}
\usepackage{colortbl}
\usepackage{arydshln}
\usepackage{mathtools}

\newcommand*{\mbold}[1]{\mbox{\boldmath $#1$}}

%\renewcommand*{\labelenumi}{(\arabic{enumi})}

\newcommand*{\transp}[1]{\prescript{t\!}{}{#1}}

\newcommand*{\grad}{{\rm grad}}
\newcommand*{\divg}{{\rm div}}
\newcommand*{\rot}{{\rm rot}}
\newcommand*{\trace}[1]{{\rm tr}\!{#1}}


\title{Formula1.21}

\begin{document}
\maketitle

\begin{abstract}
  \begin{equation}
    \rot\grad f(\mbold{x}) = \mbold{0}
  \end{equation}
\end{abstract}

\section*{本証明}
Formula1.19のlemmaを利用し, 成分計算する. 引数は省略する. 
完全反対称と, 対称の積ゆえ, 当たり前だが, あえて計算する. 

\begin{eqnarray}
  && (\rot \grad f)_i \nonumber \\
  && = \sum_{j, k} \epsilon_{ijk}\frac{\partial}{\partial x_j}\frac{\partial f}{\partial x_k} \nonumber \\
  && = \frac{1}{2} \left( \sum_{j, k} \epsilon_{ijk}\frac{\partial}{\partial x_j}\frac{\partial f}{\partial x_k} + \sum_{j, k} \epsilon_{ikj}\frac{\partial}{\partial x_k}\frac{\partial f}{\partial x_j}\right)\nonumber \\
  && = \frac{1}{2} \left( \sum_{j, k} \epsilon_{ijk}\frac{\partial}{\partial x_j}\frac{\partial f}{\partial x_k} + \sum_{j, k} (-\epsilon_{ijk})\frac{\partial}{\partial x_j}\frac{\partial f}{\partial x_k}\right)\nonumber \\
  && = \mbold{0}
\end{eqnarray}
3行目では, 全く同じものを, $j, k$の文字を入れ替えただけのもので表現し, それとの和の半分, という表現をしている. 
4行目では, Formula1.19のlemmaの5番目の式を使い, 添え字を入れ替えている. 
しれっと, 偏微分の順序を入れ替えているが, これができる条件として, 
$f_{xy}$, $f_{yx}$がともに存在し, 連続である, という要請が暗にされている. 
\url{https://manabitimes.jp/math/1174}が詳しい. 


\end{document}
