\documentclass{jsarticle} \usepackage[dvipdfmx]{graphicx} \usepackage[dvipdfmx]{hyperref}
\usepackage{amsmath}
\usepackage{color}
\usepackage{colortbl}
\usepackage{arydshln}
\usepackage{mathtools}

\newcommand*{\mbold}[1]{\mbox{\boldmath $#1$}}

%\renewcommand*{\labelenumi}{(\arabic{enumi})}

\newcommand*{\transp}[1]{\prescript{t\!}{}{#1}}

\newcommand*{\grad}{{\rm grad}}
\newcommand*{\divg}{{\rm div}}
\newcommand*{\rot}{{\rm rot}}
\newcommand*{\trace}[1]{{\rm tr}\!{#1}}


\title{Question1.31}

\begin{document}
\maketitle

\begin{abstract}
  ベクトル場$\mbold{F}(x, y, z) = (x + y, xz, y + z^2)$を, 
  単位円周 $C : (\cos t, \sin t, 0)$, 単位円板 $S_1 : x^2 + y^2 \leq 1$, 半球面 $S_2 : x^2 + y^2 + z^2 = 1$, $z\geq 0$
  で積分. 
\end{abstract}

\section*{計算}
定義通りに計算する. 早い話が, 球の8分割. 
\begin{eqnarray}
  && \frac{\partial \mbold{S}}{\partial \theta} = 
  \begin{pmatrix}
    \cos\theta \cos\phi \\
    \cos\theta \sin\phi \\
    -\sin\theta\\
  \end{pmatrix}
  \nonumber \\
  && \frac{\partial \mbold{S}}{\partial \phi} = 
  \begin{pmatrix}
    -\sin\theta \cos\phi \\
    -\sin\theta \sin\phi \\
    0 \\
  \end{pmatrix}
  \nonumber \\
  && \frac{\partial \mbold{S}}{\partial \theta} \times \frac{\partial \mbold{S}}{\partial \phi} = 
  \begin{pmatrix}
    -\sin^2\theta \cos\phi \\
    \sin^2\theta \sin\phi \\
    \cos\theta \sin\theta \\
  \end{pmatrix}
\end{eqnarray}
を使って, 積分を計算すると, 
\begin{eqnarray}
  \int_S \left| \frac{\partial \mbold{S}}{\partial \theta} \times \frac{\partial \mbold{S}}{\partial \phi} \right|d\theta d\phi = 
  \int_0^{\frac{\pi}{2}}d\theta \int_0^{\frac{\pi}{2}}d\phi \sin\theta 
  = \frac{\pi}{2}
\end{eqnarray}
と, 確かに, 球の表面積$4\pi$の8分の1になっている. 
なお, $\sin\theta$は, 球の経線の半径になっており, 積分すべき, 微少な平行四辺形が, 高緯度になればなるほど, $\phi$軸方向に細長くなっていくことを示している. 
同様に, $f$の面積分を計算すると, 
\begin{eqnarray}
  \int_S \left| \frac{\partial \mbold{S}}{\partial \theta} \times \frac{\partial \mbold{S}}{\partial \phi} \right| \cos\theta \sin\phi d\theta d\phi 
  = \frac{1}{2}
\end{eqnarray}
多分計算結果には特に意味はないと思われる. 

\end{document}
