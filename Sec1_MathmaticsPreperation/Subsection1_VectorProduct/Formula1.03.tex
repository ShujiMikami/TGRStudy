\documentclass{jsarticle}
\usepackage[dvipdfmx]{graphicx}
\usepackage{amsmath}
\usepackage{color}
\usepackage{colortbl}
\usepackage{arydshln}

\newcommand*{\mbold}[1]{\mbox{\boldmath $#1$}}

\renewcommand*{\labelenumi}{(\arabic{enumi})}

\title{Formula1.03}

\begin{document}
\maketitle
\[
  \mbold{a} \cdot (\mbold{b} \times \mbold{c}) = \mbold{b} \cdot (\mbold{c} \times \mbold{a}) = \mbold{c} \cdot (\mbold{a} \times \mbold{b})
\]
これらの式は, $\mbold{a}$, $\mbold{b}$, $\mbold{c}$で張られる平行六面体の体積
最初の式について成分表示すると, 
\[
  \mbold{a} \cdot (\mbold{b} \times \mbold{c}) = 
  \begin{pmatrix}
        a_1 \\
        a_2 \\
        a_3
  \end{pmatrix}
  \cdot
  \begin{pmatrix}
    b_2 c_3 - b_3 c_2 \\
    b_3 c_1 - b_1 c_3 \\
    b_1 c_2 - b_2 c_1
  \end{pmatrix}
  = a_1 b_2 c_3 - a_1 b_3 c_2 + a_2 b_3 c_1 - a_2 b_1 c_3 + a_3 b_1 c_2 - a_3 b_2 c_1
\]
この式について, 第二式, 第三式を実現するような, $a, b, c$の単純な文字の置き換えで, 式が変化しないかどうかを確認する. 
例えば, 第二式の場合は, $a \rightarrow b$, $b \rightarrow c$, $c \rightarrow a$という置き換えであるが, 第一項について, その置き換えをすると, $b_1 c_2 a_3 = a_3 b_1 c_2$となり, 第5項と一致する. これを, 他の項についても以下のような表を用いて確認していく. 
\begin{table}[hbtp]
  \caption{第一式の添字}
  \centering
  \begin{tabular}{|c|c|c|c|}
    \hline
    a & b & c & sgn \\
    \hline \hline
    1 & 2 & 3 & + \\
    \hdashline
    \rowcolor{yellow}1 & 3 & 2 & - \\
    \hdashline
    \rowcolor{green}2 & 3 & 1 & + \\
    \hdashline
    \rowcolor{red}2 & 1 & 3 & - \\
    \hdashline
    \rowcolor{cyan}3 & 1 & 2 & + \\
    \hdashline
    \rowcolor{magenta}3 & 2 & 1 & - \\
    \hline
  \end{tabular}
\end{table}

\begin{table}[hbtp]
  \caption{第二式に対応した入れ替えによる添字}
  \centering
  \begin{tabular}{|c|c|c|c|}
    \hline
    a & b & c & sgn \\
    \hline \hline
    \rowcolor{green}2 & 3 & 1 & + \\
    \hdashline
    \rowcolor{magenta}3 & 2 & 1 & - \\
    \hdashline
    \rowcolor{cyan}3 & 1 & 2 & + \\
    \hdashline
    \rowcolor{yellow}1 & 3 & 2 & - \\
    \hdashline
    1 & 2 & 3 & + \\
    \hdashline
    \rowcolor{red}2 & 1 & 3 & - \\
    \hline
  \end{tabular}
\end{table}

\begin{table}[hbtp]
  \caption{第三式に対応した入れ替えによる添字}
  \centering
  \begin{tabular}{|c|c|c|c|}
    \hline
    a & b & c & sgn \\
    \hline \hline
    \rowcolor{cyan}3 & 1 & 2 & + \\
    \hdashline
    \rowcolor{red}2 & 1 & 3 & - \\
    \hdashline
    1 & 2 & 3 & + \\
    \hdashline
    \rowcolor{magenta}3 & 2 & 1 & - \\
    \hdashline
    \rowcolor{green}2 & 3 & 1 & + \\
    \hdashline
    \rowcolor{yellow}1 & 3 & 2 & - \\
    \hline
  \end{tabular}
\end{table}
表を比べると, ただ入れ替えただけの結果が, 確かに符号まで含めて, 色付けしたところ同士で同じ項になっており, 恒等式になっている. 
\\

$\mbold{a} \cdot (\mbold{b} \times \mbold{c})$について考える. Theorem1.02によれば, $\mbold{b} \times \mbold{c}$は, $\mbold{b}$と$\mbold{c}$で張られる平行四辺形の面積$S$の大きさを持ち, その面に対する法線の方向$\mbold{n}$のベクトルであった. 
したがって
\[
  \mbold{a} \cdot (\mbold{b} \times \mbold{c}) = S \mbold{a} \cdot \mbold{n}
\]
右辺は, $\mbold{a}$の$\mbold{n}$方向成分の符号付き長さであり, 件の平行六面体においては, $S$を底面とするなら, 高さに相当する. 
したがって, これは体積を表す. 
\end{document}
