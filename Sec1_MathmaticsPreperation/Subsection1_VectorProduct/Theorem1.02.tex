\documentclass{jsarticle}
\usepackage[dvipdfmx]{graphicx}
\usepackage{amsmath}
\usepackage{color}
\usepackage{colortbl}
\usepackage{arydshln}

\newcommand*{\mbold}[1]{\mbox{\boldmath $#1$}}

\renewcommand*{\labelenumi}{(\arabic{enumi})}

\title{Threorem1.02}

\begin{document}
\maketitle

\begin{abstract}
  3次元ベクトル $\mbold{a}$, $\mbold{b}$に関して, 
  \begin{enumerate}
    \item $\mbold{a} \cdot \mbold{a} = \mbold{0}$
    \item $\mbold{a} \times \mbold{b} \perp \mbold{a}$, $\mbold{a} \times \mbold{b} \perp \mbold{b}$
    \item $\mbold{a}$と$\mbold{b}$で張る平行四辺形の面積$S$に対し, $S = |\mbold{a} \times \mbold{b}|$
  \end{enumerate}
  の証明. 
\end{abstract}

\begin{enumerate}
  \item $\mbold{a} \cdot \mbold{a} = \mbold{0}$ \\
    ただそのまま計算するだけ. \\
    \[
      \begin{pmatrix}
        a_1 \\
        a_2 \\
        a_3
      \end{pmatrix}
      \times
      \begin{pmatrix}
        a_1 \\
        a_2 \\
        a_3
      \end{pmatrix}
      =
      \begin{pmatrix}
        a_2 a_3 - a_3 a_2 \\
        a_3 a_1 - a_1 a_3 \\
        a_1 a_2 - a_2 a_1
      \end{pmatrix}
      = \mbold{0}
    \]
  \item $\mbold{a} \times \mbold{b} \perp \mbold{a}$, $\mbold{a} \times \mbold{b} \perp \mbold{b}$ \\
    ただそのまま計算するだけ. \\
    \[
      \begin{pmatrix}
        a_2 b_3 - a_3 b_2 \\
        a_3 b_1 - a_1 b_3 \\
        a_1 b_2 - a_2 b_1
      \end{pmatrix}
      \cdot
      \begin{pmatrix}
        a_1 \\
        a_2 \\
        a_3
      \end{pmatrix}
      = a_1 a_2 b_3 - \textcolor{red}{a_1 a_3 b_2} + \textcolor{blue}{a_2 a_3 b_1} - a_1 a_2 b_3 + \textcolor{red}{a_1 a_3 b_2} - \textcolor{blue}{a_2 a_3 b_1} = 0 
    \]
    \[
      \begin{pmatrix}
        a_2 b_3 - a_3 b_2 \\
        a_3 b_1 - a_1 b_3 \\
        a_1 b_2 - a_2 b_1
      \end{pmatrix}
      \cdot
      \begin{pmatrix}
        b_1 \\
        b_2 \\
        b_3
      \end{pmatrix}
      = b_1 a_2 b_3 - \textcolor{red}{b_1 a_3 b_2}
      + \textcolor{red}{b_2 a_3 b_1} - \textcolor{blue}{b_2 a_1 b_3}
      + \textcolor{blue}{b_3 a_1 b_2} - b_3 a_2 b_1 = 0 
    \]
  \item $\mbold{a}$と$\mbold{b}$で張る平行四辺形の面積$S$に対し, $S = |\mbold{a} \times \mbold{b}|$ \\
    $\mbold{a}$と, $\mbold{b}$の成す角度を$\theta$とすると, 
    \[
      S = |\mbold{a}||\mbold{b}||\sin\theta|
      = |\mbold{a}||\mbold{b}|\sqrt{1 - \cos^2\theta}
      = \sqrt{|\mbold{a}|^2 |\mbold{b}|^2 - (\mbold{a}\cdot\mbold{b})^2}
    \]
    成分表示すると, 
    \begin{equation}\label{eq-1}
      \sqrt{|\mbold{a}|^2 |\mbold{b}|^2 - (\mbold{a}\cdot\mbold{b})^2}
      = \sqrt{(\sum_{i = 1}^{3} a_i^2) (\sum_{i = 1}^{3} b_i^2) - (\sum_{i = 1}^{3} a_i b_i)^2}
    \end{equation}
    式(\ref{eq-1})右辺のルート内の第一項は, 多項式の展開であり, 添字が同じ項と, 異なる項にわけられる. 
    \[
      (\sum_{i = 1}^{3} a_i^2) (\sum_{i = 1}^{3} b_i^2) = \sum_{i,j=1}^{3}a_i^2 b_j^2 = \sum_{i = 1}^{3}a_i^2 b_i^2 + \sum_{i \neq j}a_i^2 b_j^2
    \]
    第二項も同様に展開でき, 
    \[
      (\sum_{i = 1}^{3} a_i b_i)^2 = \sum_{i,j = 1}^{3}a_i b_i a_j b_j = \sum_{i = 1}^{3}a_i^2 b_i^2 + \sum_{i\neq j}a_i a_j b_i b_j
    \]
    以上により, 
    \begin{equation}\label{eq-2}
      S^2 = \sum_{i \neq j}a_i^2 b_j^2 - \sum_{i \neq j}a_i a_j b_i b_j
    \end{equation}
    式(\ref{eq-2})第一項, 第二項ともに, $i > j$のときと, $i < j$のときに分けることができ, 
    \begin{equation}\label{eq-3}
      \sum_{i \neq j}a_i^2 b_j^2 - \sum_{i \neq j}a_i a_j b_i b_j
      = \sum_{i > j}a_i^2 b_j^2 + \sum_{i < j}a_i^2 b_j^2 - \sum_{i > j}a_i a_j b_i b_j - \sum_{i < j}a_i a_j b_i b_j
    \end{equation}
    式(\ref{eq-3})右辺の各4項のインデックス$i, j$は, 項の中での計算でクローズしているものであり, 文字を書き換えたとしても, 計算結果に影響を与えない. そこで, 第二項, 及び, 第四項について, $i \leftrightarrow j$の入れ替えを実施し, 
    \begin{equation}\label{eq-4}
      \sum_{i > j}a_i^2 b_j^2 + \sum_{i < j}a_i^2 b_j^2 - \sum_{i > j}a_i a_j b_i b_j - \sum_{i < j}a_i a_j b_i b_j
      = \sum_{i > j}a_i^2 b_j^2 + \sum_{i > j}a_j^2 b_i^2 - \sum_{i > j}a_i a_j b_i b_j - \sum_{i > j}a_j a_i b_j b_i
    \end{equation}
    を得る. 式(\ref{eq-4})右辺のインデックス$i, j$, 及び和条件$i > j$が揃っているので, この和を一つにまとめると, 
    \begin{equation}\label{eq-5}
      \sum_{i > j}a_i^2 b_j^2 + a_j^2 b_i^2 - a_i a_j b_i b_j - a_j a_i b_j b_i = \sum_{i > j}a_i^2 b_j^2 + a_j^2 b_i^2 - 2 a_i a_j b_i b_j = \sum_{i > j}(a_i b_j - a_j b_i)^2
    \end{equation}
    となる. 
    式(\ref{eq-5})最右辺の和の中の, $i = 3, j = 2$を考えると, $(a_3 b_2 - a_2 b_3)^2 = (a_2 b_3 - a_3 b_2)^2 = (\mbold{a} \times \mbold{b})_z^2$となっており, 同様に, $i = 3, j = 1$で$y$成分, $i = 2, j = 1$で, $x$成分になっているため, これは, $|\mbold{a} \times \mbold{b}|^2$にほかならない. 
\end{enumerate}

\end{document}
