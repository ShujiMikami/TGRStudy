\documentclass{jsarticle}
\usepackage[dvipdfmx]{graphicx}
\usepackage{amsmath}
\usepackage{color}
\usepackage{colortbl}
\usepackage{arydshln}
\usepackage{mathtools}

\newcommand*{\mbold}[1]{\mbox{\boldmath $#1$}}

\renewcommand*{\labelenumi}{(\arabic{enumi})}

\newcommand*{\transp}[1]{\prescript{t\!}{}{#1}}

\newcommand*{\grad}{{\rm grad}}

\title{Theorem1.12}

\begin{document}
\maketitle

\begin{abstract}
  二次元スカラー場$f(x, y)$
  \begin{enumerate}
    \item $\grad f$は等高線と直交 \label{item-1}
    \item $\grad f$は傾きが最大の方向, $|\grad f |$は傾きの最大値
  \end{enumerate}
\end{abstract}

\section*{lemma}
等高線とは, $f(x, y)$が一定値$z_0$を取る, $(x, y)$の集合に対し, $(x, y, z_0)$が描く曲線のことである. 
ある$z_0$が与えられたとき, $f(x, y)$が$C^1$級であるならば, $C^1$級の関数により, 
\begin{subequations}
  \begin{eqnarray}
    x = x(t) \\
    y = y(t)
  \end{eqnarray}
\end{subequations}
と表すことができる. 

\section*{本証明}
\subsection*{"$\grad f$は等高線と直交"の証明}
\[
  F(x, y) \equiv f(x, y) - z_0
\]
とすると, 
\[
  F(x, y) = 0
\]
なので, $F(x, y)$に対し, 陰関数定理により, $F_y(x, y) \neq 0$ならば, $(x, y)$の十分近傍で, $y = y(x)$なる$C^1$級の$y(x)$が存在する. 
または, $F_y(x, y) = 0$であっても, $F_x(x, y) \neq 0$ならば, $x = x(y)$なる$C^1$級の$x(y)$が存在する. 
どちらのケースでも, 議論は対称なので, 今は, $F_y(x, y) \neq 0$のケースで議論をすすめる. 

$F$の変化は, $x$の変化として表現でき, Theorem1.07を利用すると, 
\begin{equation}\label{eq-1}
  \frac{dF(x, y(x))}{dx} = \frac{\partial F}{\partial x}\frac{dx}{dx} + \frac{\partial F}{\partial y}\frac{dy}{dx}
  = \frac{\partial F}{\partial x} + \frac{\partial F}{\partial y}\frac{dy}{dx} = 0
\end{equation}
$\grad$の定義, および, 陰関数定理の前提条件として, $F_y(x, y) \neq 0$より, 式(\ref{eq-1})は, 
\begin{equation}
  (\grad F) 
  \cdot
  \begin{pmatrix}
    1 \\
    \frac{dy}{dx} \\
  \end{pmatrix}
  =
  (\grad f) 
  \cdot
  \begin{pmatrix}
    1 \\
    \frac{dy}{dx} \\
  \end{pmatrix}
  = 0
\end{equation}
ベクトル$(1, dy/dx)$は, $f(x, y) = z_0$を満たす, $y = y(x)$の, $x$における接線に沿って, $x$軸方向に$1$進むベクトルであり,
それとの内積が$0$なので, 等高線と勾配は直交する. 



\end{document}
