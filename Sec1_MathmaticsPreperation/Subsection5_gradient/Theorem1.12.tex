\documentclass{jsarticle}
\usepackage[dvipdfmx]{graphicx}
\usepackage{amsmath}
\usepackage{amssymb}
\usepackage{color}
\usepackage{colortbl}
\usepackage{arydshln}
\usepackage{mathtools}

\newcommand*{\mbold}[1]{\mbox{\boldmath $#1$}}

\renewcommand*{\labelenumi}{(\arabic{enumi})}

\newcommand*{\transp}[1]{\prescript{t\!}{}{#1}}

\newcommand*{\grad}{{\rm grad}}

\title{Theorem1.12}

\begin{document}
\maketitle

\begin{abstract}
  二次元スカラー場$f(x, y)$
  \begin{enumerate}
    \item $\grad f$は等高線と直交 \label{item-1}
    \item $\grad f$は傾きが最大の方向, $|\grad f |$は傾きの最大値
  \end{enumerate}
\end{abstract}

\section*{lemma}
等高線とは, $f(x, y)$が一定値$z_0$を取る, $(x, y)$の集合に対し, $(x, y, z_0)$が描く曲線のことである. 
ある$z_0$が与えられたとき, $f(x, y)$が$C^1$級であり, かつ, $f$の勾配がゼロベクトルでないならば, $C^1$級の関数により, 
\begin{subequations}
  \begin{eqnarray}
    x = \phi(t) \\
    y = \psi(t)
  \end{eqnarray}
\end{subequations}
と表すことができる. 

\subsection*{証明}
勾配がゼロベクトルではない条件を, $\partial f / \partial x \neq 0$か, $\partial f / \partial y \neq 0$のうち, 少なくともどちらかが成立する, と読み替える. 同じことである. 
まずは, $\partial f / \partial y \neq 0$とする. 
このとき, 陰関数定理により, $f(x, y) = f(x, \psi(x))$なる, $\psi(x)$が存在する. 
シンプルに, $x = \phi(t) \equiv t$とすれば, $y = \psi(\phi(t)) = \psi(t)$となり, 存在の証明はされた. 

もう少し拡張性をもった言い方をすれば, 等高線上の$x$を全て含む集合$V \subset \mathbb{R}$に対し, 
$C^1$級の関数で定義される全射$\phi : \mathbb{R} \rightarrow V$を定義したらならば, 
陰関数定理により, $C^1$級の$\psi$が存在し, $y = \psi(\phi(t))$であり, 微分の連鎖率から, $\psi(\phi(t)) \equiv \Psi(t)$も$C^1$級である. 

この議論は, $\partial f / \partial x \neq 0$の場合と対称であり, 同様に成立する. 
従って, 少なくとも一つのパラメータ表示の存在が証明され, それ以外にも, $C^1$級のパラメータ表示が$x$について見つかったならば, $y$も$C^1$級のパラメータ表示関数が存在する. 

\section*{本証明}
\subsection*{"$\grad f$は等高線と直交"の証明}
\subsubsection*{勾配がゼロベクトルとなる場合}
この場合は, そもそもの直交の定義によるところであるが, ベクトルの直交とは, 内積がゼロであることだ, と定義するならば, 自明に成立している. これは, すなわち, ゼロベクトル, 点は, 全てのベクトルと直交する, というスタンスの定義である. こう言い切る参考資料もある. 
一方で, ベクトルの直交とは, ベクトル同士のなす角が90度であることだ, という定義をするならば, 勾配がゼロな領域では, 接線が一意に定義できないため, 同じ面内の全てのベクトルは, 接線と直交する, という言い方になる. 
すなわち, 勾配が0になる点付近での展開, 
\begin{equation}
  f(x + \Delta x, y + \Delta y) - f(x, y) = \frac{\partial f}{\partial x}\Delta x + \frac{\partial f}{\partial y}\Delta y + o(x) + o(y)
\end{equation}
は, $\Delta x$, $\Delta y$がどのような形で近づこうとも, 必ずゼロに収束する. つまり, $(x, y)$平面内のどの方向にも接線がある, ということになる. 
これは, 凸関数の極点などと考えるとわかりやすいが, $f(x, y) = x^3$といった, 一見, 等高線が一つしか無いように思える関数についても, 実は, $x = 0$付近では, いかなる方向へ向いていても接線となっていることがわかる. 
よって, この定義であっても, 勾配がゼロベクトルの場合は, 等高線と直交する. 
\subsubsection*{勾配がゼロベクトルではない場合}
lemmaにより, $f(x, y) = f(x(t), y(t)$と媒介変数表示でき, 媒介変数の関数は, $C^1$級なので, 
\begin{equation}
  \frac{df}{dt} = \frac{\partial f}{\partial x}\frac{dx}{dt} + \frac{\partial f}{\partial y}\frac{dy}{dt} = 0
\end{equation}
なので, $\grad f$と, 媒介変数表示された曲線の接線方向ベクトルが直交すると言える. 

ある点からの近傍への展開, 
\begin{equation}
  f(x + \Delta x, y + \Delta y) - f(x, y) = \frac{\partial f}{\partial x}\Delta x + \frac{\partial f}{\partial y}\Delta y + o(x) + o(y)
  = (\grad f)\cdot (\Delta x, \Delta y)
\end{equation}
は, ある点からの移動方向と, 勾配の内積で表され, この値が最大の値を持つときは, 移動方向と, 勾配の方向が同じときである. すなわち, 勾配の方向が, $z$軸方向への最大変化をもたらす方向であり, 傾斜がもっともきつい方向である. 
また, 内積を, 図形的な表現をすると, 
\[
  | \grad f ||(\Delta x, \Delta y)|\cos\theta
\]
であり, もっともきつい方向への傾斜をとり, $(\Delta x, \Delta y)$の大きさを$1$に取れば, $|\grad f|$が, 単位移動量あたりの, 高さ($z$)方向の変位量, すなわち, 傾斜の大きさであることがわかる. 


\end{document}
