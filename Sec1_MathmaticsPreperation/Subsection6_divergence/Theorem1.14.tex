\documentclass{jsarticle}
\usepackage[dvipdfmx]{graphicx}
\usepackage{amsmath}
\usepackage{color}
\usepackage{colortbl}
\usepackage{arydshln}
\usepackage{mathtools}

\newcommand*{\mbold}[1]{\mbox{\boldmath $#1$}}

%\renewcommand*{\labelenumi}{(\arabic{enumi})}

\newcommand*{\transp}[1]{\prescript{t\!}{}{#1}}

\newcommand*{\grad}{{\rm grad}}
\newcommand*{\divg}{{\rm div}}

\newcommand*{\trace}[1]{{\rm tr}\!{#1}}

\title{Theorem1.14}

\begin{document}
\maketitle

\begin{abstract}
  直交座標系$\Sigma_{o}$, $\Sigma_{o}^\prime$に対し, 
  ある3次元ベクトル場が, 
  \begin{subequations}
    \begin{eqnarray}
      && \mbold{A} = (A_x(x, y, z), A_y(x, y, z), A_z(x, y, z)) 
      ~~~ at ~~ \Sigma_{o} \\
      && \mbold{A^\prime} = (A^\prime_x(x^\prime, y^\prime, z^\prime), 
      A^\prime_y(x^\prime, y^\prime, z^\prime), 
      A^\prime_z(x^\prime, y^\prime, z^\prime)) 
      ~~~ at ~~ \Sigma^\prime_{o}
    \end{eqnarray}
  \end{subequations}
  と表されるとき, 
  \begin{equation}\label{eq-2}
    \frac{\partial A_x}{\partial x}
    + \frac{\partial A_y}{\partial y}
    + \frac{\partial A_z}{\partial z}
    = \frac{\partial A_x^\prime}{\partial x^\prime}
    +\frac{\partial A_y^\prime}{\partial y^\prime}
    +\frac{\partial A_z^\prime}{\partial z^\prime}
  \end{equation}
  すなわち, 発散は, 直交座標のとり方によらない. 
\end{abstract}

\section{本証明}
直交座標系の属性は, 
\begin{itemize}
  \item 原点
  \item 基底ベクトル
\end{itemize}
であり, 原点の差異は並進, 基底ベクトルの差異は, 直交変換により, 表現されるので, それぞれの場合に分けて考えていく. 

\subsection{並進}
並進変換は, 
\begin{subequations}
  \begin{eqnarray}
    x^\prime = x + \Delta x \\
    y^\prime = y + \Delta y \\
    z^\prime = z + \Delta z
  \end{eqnarray}
\end{subequations}
で表される, 原点のみの移動である. このとき, 基底ベクトルは, 変化しないので, 結局各成分について, 
\begin{equation}\label{eq-4}
  A_i(x, y, z) = A^\prime_i(x^\prime, y^\prime, z^\prime). ~~ i = x, y, z
\end{equation}
が成立する.

並進変換の関係は全射になっているので, $x^\prime(x, y, z) = x + \Delta x$のような媒介変数表示の関係であると捉えることができる. 
したがって, $A_x^\prime(x^\prime, y^\prime, z^\prime)$に対する, $x$の偏微分を考えることができ, 式(\ref{eq-4})から, 
\begin{equation}\label{eq-5}
  \frac{\partial A_x(x, y, z)}{\partial x} = \frac{\partial A^\prime_x(x^\prime(x, y, z), y^\prime(x, y, z), z^\prime(x, y, z))}{\partial x}. 
\end{equation}

Theorem1.07を利用すると, 式(\ref{eq-5})は, 
\begin{equation}\label{eq-6}
  \frac{\partial A^\prime_x(x^\prime(x, y, z), y^\prime(x, y, z), z^\prime(x, y, z))}{\partial x}
  = \frac{\partial A^\prime_x}{\partial x^\prime}\frac{\partial x^\prime}{\partial x}
  + \frac{\partial A^\prime_x}{\partial y^\prime}\frac{\partial y^\prime}{\partial x}
  + \frac{\partial A^\prime_x}{\partial z^\prime}\frac{\partial y^\prime}{\partial x}
  = \frac{\partial A_x^\prime}{\partial x^\prime}
\end{equation}
よって, 
\begin{equation}
  \frac{\partial A^\prime_x}{\partial x^\prime} = \frac{\partial A_x}{\partial x}
\end{equation}
他の軸についても同様の議論をすれば, 式(\ref{eq-2})が成立する. 

\subsection{直交変換}
直交変換は, 
\begin{equation}
  \begin{pmatrix}
    x^\prime \\
    y^\prime \\
    z^\prime
  \end{pmatrix}
  = U 
  \begin{pmatrix}
    x \\
    y \\
    z 
  \end{pmatrix}.
  ~~
  \transp{U} U = E
\end{equation}
である. 
$x$成分だけを表現すると, 
\begin{equation}
  x^\prime = U_{11}x + U_{12}y + U_{13}z.
\end{equation}
この変換も空間全射になっており, Theorem1.08により, $U$の逆行列が存在することから, 全単射になっている. また, 逆行列と転置行列が等しいので,  
\begin{equation}\label{eq-11}
  x = \transp{U}_{11}x^\prime + \transp{U}_{12}y^\prime + \transp{U}_{13}z^\prime.
\end{equation}
も, 全単射である. 

ここで, 
\begin{subequations}
  \begin{eqnarray}
    && V_{ij} \equiv \frac{\partial A_i(x, y, z)}{\partial x_j} \\
    && W_{ij} \equiv \frac{\partial A^\prime_i(x^\prime, y^\prime, z^\prime)}{\partial x^\prime_j}
  \end{eqnarray}
\end{subequations}
と定義すると, 
式(\ref{eq-2})の左辺は, $\trace{V}$, 右辺は, $\trace{W}$となる. 

座標変換において, $\mbold{A}$も同じ変換を受けるので, 
\begin{equation}\label{eq-12}
  W_{ij} = \frac{\partial}{\partial x^\prime_j}\sum_k U_{ik}A_k
  = \sum_k U_{ik}\frac{\partial A_k}{\partial x^\prime_j}.
\end{equation}

式(\ref{eq-11})も全単射であることから, 式(\ref{eq-12})中の$A_k$の引数である, $x$, $y$, $z$は, $x^\prime$, $y^\prime$, $z^\prime$によるパラメータ表示を$C^1$級で行うことができ, 
Theorem1.07を適用すれば, 
\begin{equation}\label{eq-13}
  \sum_k U_{ik}\frac{\partial A_k}{\partial x^\prime_j}
  = \sum_k U_{ik} \sum_l \frac{\partial A_k}{\partial x_l}\frac{\partial x_l}{\partial x^\prime_j}
\end{equation} 
式(\ref{eq-11})を利用すると, 
\begin{equation}
  \frac{\partial x_l}{\partial x^\prime_j} = \sum_m \transp{U}_{lm}\frac{\partial x^\prime_m}{\partial x^\prime_j}
\end{equation}
であるが, 右辺の微分項は, $m = j$のときは$1$で, それ以外は$0$になるので, $m$についての和の結果は, $\transp{U}_{lj}$になる. 
よって, 式(\ref{eq-13})は, 
\begin{equation}
  \sum_k U_{ik} \frac{\partial A_k}{\partial x^\prime_j} 
  = \sum_k U_{ik} \sum_l V_{kl} \transp{U}_{lj}
  = (U V \transp{U})_{ij}
\end{equation}
つまり, 行列として, 
\begin{equation}
  W = UV\transp{U}.
\end{equation}
行列のトレースの性質と, Theorem1.08を利用し, 
\begin{equation}
  \trace{W} = \trace{(UV\transp{U})} = \trace{(\transp{U} UV)} = \trace{V}
\end{equation}
により, 式(\ref{eq-2})が成り立つ. 

\end{document}
